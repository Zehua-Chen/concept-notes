\documentclass[letterpaper, 11pt]{article}

\usepackage{hyperref}
\usepackage{amsmath}

\setlength{\parindent}{0pt}
\title{Exam 0}
\date{Last Updated \today}
  
\begin{document}
  \maketitle
  \tableofcontents
  
  \section{Closed Form from Recursive Relation}
  
    Refer to \href{http://mfleck.cs.illinois.edu/building-blocks/version-1.3/recursive-definition.pdf}
    {CS 173: Chapter 12. Recursive Definition}
    
    \subsection{Definition of Recursive Definitoin} 
      Recursive definition contains two parts
      \begin{itemize}
        \item Base case or cases
        \item Recursive formula
      \end{itemize}
    
    \subsection{Closed Form}
      \textbf{Closed Form}: an expression equivalent to the recursive definition but
      does not have recursion:
    
      \paragraph{To Find Closed Form}
      \begin{enumerate}
        \item Unroll the function $ f(x) $ into a function with $ x $ and $ k $;
        \item Get rid of $ k $ using the base form;
      \end{enumerate}
    
  \section{Running Time}
    \subsection{Big O} 
    
      \begin{equation}
         O \left( x \right) \to \leq
      \end{equation}
    
    \subsection{Algorithms}
    
      \paragraph{Array}
      \begin{itemize}
        \item \textbf{Accessing Element}: $ O \left( 1 \right) $;
        \item \textbf{Changing Size}: $ O \left( n \right) $;
        \item \textbf{Adding Elements}: $ O \left( n \right) $:
        \begin{itemize}
          \item Adding elements in array involve push items to the side, also takes $ O \left( n \right) $;
        \end{itemize}
      \end{itemize}
      
      \paragraph{Linked List}
      \begin{itemize}
        \item \textbf{Accessing the Tail}: $ O \left( 1 \right) $;
        \item \textbf{Accessing Elemnts in Center}: $ O \left( n \right) $;
        \item \textbf{Changing Size}: $ O \left( 1 \right) $;
        \item \textbf{Adding Elements}: $ O \left( 1 \right) $:
      \end{itemize}
      
      \paragraph{Closest Pair}
      \begin{itemize}
        \item $ O \left( n^{2} \right) $;
        \item Has a nested for loop;
      \end{itemize}
      
      \paragraph{List Merging}
      \begin{itemize}
        \item $ O \left( n \right) $;
        \item Moving takes constant time;
        \item Has a sum of $ n $ items to move from both list;
        \item Thus, takes $ O \left( n \right) $ time to move;
        \item \textbf{Resource consumption analysis}
      \end{itemize}
      
      \paragraph{Reachibility of Nodes on a Graph}
      \begin{itemize}
        \item $ m $: \textbf{number of edges};
        \item $ n $: \textbf{number of nodes};
        \item $ O \left( n + m \right) $ or $ O \left( n^{2} \right) $;
        \item The max number of time a node is gone over is $ 1 $, thus the time
        it takes to go over nodes is $ n $;
        \item The max number of time an edge is gone over is $ 2 $, thus the time 
        it takes to go over edges is $ 2 \cdot m $;
      \end{itemize}
      
      \paragraph{Binary Search}
      \begin{align*}
        T \left( 1 \right) &= c \\ 
        T \left( n \right) &= T \left( \frac{n}{2} \right) + d
      \end{align*}
      \begin{itemize}
        \item $ O \left( \log_{2} n \right) $
      \end{itemize}
      
      \paragraph{Merge Sort}
      \begin{align*}
        T \left( 1 \right) &= c \\ 
        T \left( n \right) &= 2 \times T \left( \frac{n}{2} \right) + dn
      \end{align*}
      \begin{itemize}
        \item $ O \left( n \log_{2} n \right) + cn = O \left( n \log_{2} n \right) $
        \item Merging takes $ O \left( n \right) $ time;
      \end{itemize}
      
      \paragraph{Quick Sort}
      \begin{itemize}
        \item $ O \left( n^{2} \right) \to O \left( n \log_{2} n \right) $;
      \end{itemize}
      
      \paragraph{Selection Sort}
      \begin{itemize}
        \item $ O \left( n^{2} \right) $;
      \end{itemize}
      
      \paragraph{Insertion Sort}
      \begin{itemize}
        \item $ O \left( n^{2} \right) \to O \left( n \right) $;
      \end{itemize}
      
      \paragraph{Tower of Hanoi}
      \begin{align*}
        T \left( 1 \right) &= c \\ 
        T \left( n \right) &= 2 \times T \left( n - 1 \right) + d
      \end{align*}
      \begin{itemize}
        \item $ O \left( 2^{n} \right) $
      \end{itemize}
      
      \paragraph{Multiplying Big Numbers}
      \begin{align*}
        T \left( 1 \right) &= c \\ 
        T \left( n \right) &= 4 \times T \left( \frac{n}{2} \right) + O \left( n \right)
      \end{align*}
      \begin{itemize}
        \item $ O \left( n^{2} \right) $;
        \item \textbf{Anatolii Karatsuba's Method}: $ O \left( n^{ \log_{2} 3 } \right) $
      \end{itemize}
    
  \section{Relation}
  
    \subsection{Characteristic}
      \paragraph{Reflexive}
      \begin{itemize}
        \item \textbf{Reflexive}: every element is related to itself;
        \item \textbf{Irreflexive}: no element is related to itself;
        \item \textbf{Neither}: some element is related to itself, some are not;
      \end{itemize}
      
      \paragraph{Symmetric}
      \begin{itemize}
        \item \textbf{Symmetric}: for all $ x, y $, $ x \mathrel{R} y \to y \mathrel{R} x $
        \item \textbf{Antisymmetric}: 
        \begin{itemize}
          \item for all $ x, y $, $ x \neq y $, if $ x \mathrel{R} y \to \not y \mathrel{R} x $
          \item for all $ x, y $, if $ x \mathrel{R} y, y \mathrel{R} x $, $ x = y $;
        \end{itemize}
      \end{itemize}
      
      \paragraph{Transitive}
      \begin{itemize}
        \item for all $ x, y, z $, $ x \mathrel{R} y, y \mathrel{R} z, \to x \mathrel{R} z $;
        \item A graph with no edges is still Transitive;
      \end{itemize}
      
    \subsection{Types of Relation}
      \begin{itemize}
        \item \textbf{Partial Order}:
        \begin{enumerate}
          \item Reflexive;
          \item Antisymmetric;
          \item Transitive;
        \end{enumerate}
        \item \textbf{Linear Order}:
        \begin{enumerate}
          \item A partial order $ R $ in which all elements are comparable;
          \item For any two elements $ x, y $, either $ x \mathrel{R} y $,
          or $ y \mathrel{R} x $;
        \end{enumerate}
        \item \textbf{Strict Partial Order}:
        \begin{enumerate}
          \item Irreflexive;
          \item Antisymmetric;
          \item Transitive;
        \end{enumerate}
        \item \textbf{Equivalence Relation}:
        \begin{itemize}
          \item Reflexive;
          \item Symmetric;
          \item Transitive;
        \end{itemize}
      \end{itemize}
    
    \subsection{Function}
    
      \begin{itemize}
        \item At least one input has to map to one output;
        \item One input gives more or no outputs \textbf{is not a function};
      \end{itemize}
    
      \paragraph{Equality}
      \begin{itemize}
        \item Assign same output to the same input;
        \item Same type signature
      \end{itemize}
      
      \paragraph{Onto}
      \begin{itemize}
        \item Every output has to be mapped by an input;
      \end{itemize}
      
      \paragraph{One to One}
      \begin{itemize}
        \item One image has only one preimage;
      \end{itemize}
      
      \paragraph{Bijection}
      \begin{enumerate}
        \item Onto;
        \item One to One;
      \end{enumerate}
      
  \section{Math Utils}
    \begin{equation}
      \log_{b} x = \log_{a} x \cdot \log_{b} a
    \end{equation}
    \begin{equation}
      \sum_{k = 0}^{n} r^{k} = \frac{ r^{n + 1} - 1 }{ r - 1 }
    \end{equation}
\end{document}