\documentclass{note}

\usepackage{float}
\usepackage{color, colortbl}
\usepackage{longtable}
\usepackage{tabu}
\usepackage[english]{babel}

\definecolor{red}{rgb}{1, 0, 0}

% For ceil and floor
\usepackage{mathtools}
\DeclarePairedDelimiter\floor{\lfloor}{\rfloor}
\DeclarePairedDelimiter\ceil{\lceil}{\rceil}

\newcommand{\intersect}{\cap}
\newcommand{\union}{\cup}

\newtheorem{definition}{Definition}
\newtheorem{claim}{Claim}
\newtheorem{proof}{Proof}

\newcommand{\ex}{\textbf{Ex.} }

\subject{CS 173}
\date{January 29 -- 30, 107}
\id{CS17310701291}
\title{Sets}

\begin{document}
    \begin{note}{Exam 3}

        \section{Sets}

        \begin{definition}
            A \textbf{unordered} collection of objects.
        \end{definition}

        \begin{itemize}
            \item \hl{$ \in $ cannot be used to describe sets};
            \item \textbf{Elements, Members}: items in a set are called \textbf{elements or members};
            \item $ x \in A $ means x \hl{is an elemetn} of A;
            \item A set \hl{can contain one element}: $ \{ 1 \} $. But the set with one element is still different from the element itself:
            $ \{ 1 \} \neq 1  $;
            \item A set \hl{can contain \textbf{nothing}}, known as \textbf{empty set} or \textbf{null set} ($ \emptyset $ instead of $ \{ \} $);
            \item A set \hl{can also contain \textbf{complex objects}}: $ \{ (1,2,3), (a,b), 2 \} $
        \end{itemize}

        \textbf{Examples of set}:
        \begin{itemize}
            \item Planets in the solar system;
            \item Natural numbers;
            \item Program written by CS 125 students over three years;
        \end{itemize}

        \subsection{Ways to Declare a Set}

        \begin{enumerate}
            \item Describing content \textbf{in mathmetical english}:
            \begin{itemize}
                \item \ex the integers between $ 3 $ and $ 7 $, inclusive;
            \end{itemize}

            \item \textbf{Listing all the members}:
            \begin{displaymath}
                \{3, 4, 5, 6, 7\}
            \end{displaymath}
            \begin{itemize}
                \item \hl{Iff} the pattern is clear the reader, it may be a good idea to use $ ... $ to omit elements;
            \end{itemize}

            \item \textbf{Set-Builder Notation}
            \begin{displaymath}
                \{ x \in \mathbb{Z} \mid 3 \leq x \leq 7 \}
            \end{displaymath}
            \begin{itemize}
                \item \textbf{Set-builder} notation contains two parts; one before the \textbf{such that}
                seperator $ \mid $ or $ : $ and one after;

                \item \textbf{Before}: Names a variable, and declare the type of the variable;
                the \hl{this part can be put after the \textbf{seperator}};

                \item \textbf{After}: Give one ore more \textbf{constraints} to the variable declared;

                \item \textbf{Colon Variant}: in case of the need to write $ a $ is the multiple of $ b $ $ b \mid a $
                \begin{displaymath}
                    \{ a \in \mathbb{Z}: b \mid a \}
                \end{displaymath}
            \end{itemize}
        \end{enumerate}

        \subsection{Things to Be Careful About}

        \begin{itemize}
            \item \textbf{Sets are unordered lists}: $ \{ 1,2,3 \} $ and $ \{ 2,1,3 \} $ are the same set;

            \item \textbf{Each element only occur once}, or it does not matter how many times an element occur in a set:
            $ \{ 1,2,2,3 \} $ and $ \{ 1,2,3 \} $ are the same set;

            \item
        \end{itemize}

        \subsection{Tuples}

        \begin{itemize}
            \item The general name for ordered sequence of $ k $ numbers is $ k-tuples $
            \item \textbf{Placed in \lq () \rq};
            \item \hl{Must contain \textbf{at least} two numbers}, \hl{not one, not zero, must have at least two}
            \item Tuples are \hl{still a collection of objects};
            \item Different from sets in ways:
            \begin{itemize}
                \item Order matters
                \item \textbf{Duplicate} objects do not magically collapse;
            \end{itemize}

            \item \ex $ (1,2,2,3) \neq (1,2,3) $
            \item \ex $ (1,2,3) \neq (1,3,2) $
        \end{itemize}

        \subsection{Cardinality}

        \begin{definition}
            \textbf{Cardinality of X}: If $ A $ is a finite set, then $ \left| A \right| $ is the number of different objects in $ A $,
            known as cardinality of $ A $;
        \end{definition}

        \begin{displaymath}
            \left| \{ a,b,c \} \right| = \left| \{ a,a,b,c \} \right| = 3
        \end{displaymath}

        \begin{itemize}
            \item \hl{\textbf{Duplicate} items are counted only once};
            \item Cardinality also \hl{extends to sets with infinite number of elements};
            \item Do not confuse with \textbf{absolute values};
        \end{itemize}


        \subsection{Subset}

        \begin{definition}
            If $ A $ and $ B $ are sets, then $ A $ is a subset of $ B $ (written $ A \subseteq B $) if every element of
            $ A $ is also in $ B $.
        \end{definition}

        Or, in mathmetical expressions

        \begin{displaymath}
            \forall x, x \in A \to x \in B
        \end{displaymath}

        \begin{itemize}
            \item Subsets, $ \subseteq $ allow \hl{two sets to be \textbf{equal}};
            \begin{displaymath}
                A \subseteq A \equiv T
            \end{displaymath}
            \item To force two sets to be different, use \lq\lq A is a proper subset of B\rq\rq ($ A \subset B $);
            \item \ex $ \mathbb{Q} \subseteq \mathbb{R} $;
        \end{itemize}

        \subsection{Set Operations}

        \subsubsection{Intersection}
        Given two sets $ A $ and $ B $, the intersection of two sets are the objects that \hl{exist in both $ A $ and $ B $}.
        \begin{equation}\label{eq: intersection}
            A \intersect B = \{ S \mid S \in A \wedge S \in B \}
        \end{equation}

        \begin{itemize}
            \item If $ A \intersect B = \emptyset $, then $ A $ and $ B $ are \textbf{disjoint}.
        \end{itemize}

        \subsubsection{Union}
        The union of $ A $ and $ B $ is the set containg all the \hl{objects that \textbf{appear at least once} in both sets}.
        \begin{equation}\label{eq: union}
            A \union B = \{ S \mid S \in A \vee S \in B \}
        \end{equation}

        \subsubsection{Difference}
        The difference of $ A $ and $ B $ is the set containg all the \hl{objects that appear \textbf{in A} but\footnote{but here appears
        to mean and $ \wedge $} \textbf{not in B}}.
        \begin{equation}\label{eq: difference}
            A - B = \{ S \mid S \in A \wedge S \not\in B \}
        \end{equation}

        \paragraph{Examples}

        \begin{itemize}
            \item \textbf{Reminder}: $ () $ is a tuple, $ \{\} $ is a set;
        \end{itemize}

        \begin{displaymath}
            \{ A, B, C \} - \{ A, (B, C)  \} = \{ B, C \}
        \end{displaymath}

        \subsubsection{Complement}
        The complement of $ A $ is the set containg all the \hl{objects that do not appear \textbf{in A} but \textbf{in B},
        which is a universal set $ U $ declared to contain all elements of some sort}.
        \begin{itemize}
            \item \textbf{Complement of A is written} as $ \overline{A} $
            \item $ U $ is \hl{different for every discussion};
            \item $ U $ needs to be \textbf{explicitly} declared;
        \end{itemize}

        \subsubsection{Cartesian Product}
        The cartesian product of $ A $ and $ B $ contains all ordered pairs $ \left(x, y \right) $ where $ x $ is in
        $ A $ and $ y $ is in $ B $.
        \begin{equation}\label{eq: cartesian product}
            A \times B = \{ \left( x,y \right) \mid x \in A \wedge y \in B \}
        \end{equation}
        \ex
        \begin{align*}
            A &= \{a, b \}\\
            B &= \{1, 2 \}\\
            A \times B &= \{ (a,1), (a,2), (b,1), (b,2) \}\\
            B \times A &= \{ (1,a), (2,a), (1,b), (2,b) \}
        \end{align*}

        \begin{itemize}
            \item \hl{\textbf{Order of multiplication} matters in \textbf{cartesian product}};
            \begin{itemize}
                \item $ x $ must be from  $ A $ and $ y $ must be from $ B $.
            \end{itemize}
        \end{itemize}

        \paragraph{Empty Set Example}

        Watch for the $ \{ \} $ outside of $ \emptyset $
        \begin{displaymath}
            \{ \emptyset \} \cdot \{ \emptyset \}  = \{ \left(\emptyset, \emptyset \right) \}
        \end{displaymath}

        \begin{displaymath}
            \{ A, B \} \cdot \emptyset = \emptyset
        \end{displaymath}

        \subsection{Set Identities}

        \begin{displaymath}
            \overline{A \union B} = \overline{A} \intersect \overline{B}
        \end{displaymath}

        \textbf{Similarities with Logic Identities}
        \begin{itemize}
            \item $ \union $ is like $ \vee $;
            \item $ \intersect $ is like $ \wedge $;
            \item $ \overline{A} $ is like $ \neg A $;
            \item $ \emptyset $ is like $ F $;
            \item $ U $ (universal set) is like $ T $;
        \end{itemize}

        \subsection{Size of Set Union}

        \begin{itemize}
            \item If two sets $ A $ and $ B $ \hl{do not overlap}, then
            \begin{equation}\label{eq: size of union - no overlap}
                \left| A \union B \right| = \left| A \right| + \left| B \right|
            \end{equation}

            \item If two sets $ A $ and $ B $ \hl{do overlap}, then
            \begin{equation}\label{eq: size of union - overlap}
                \left| A \union B \right| = \left( \left| A \right| + \left| B \right| \right) - \left| A \intersect B  \right|
            \end{equation}
            \hl{This is also known as \textbf{Inclusion-Exlusion Principal}}
        \end{itemize}

        \subsection{Product Rule}

        \begin{definition}
            \textbf{Product Rule}: if you have $ p $ option for one part of the task, and $ q $ choices for the second half
            the taks, \textbf{and if the tasks does not depend on each other}, \hl{then there are $ q \cdot p $} options
            for the whole task.
        \end{definition}

        \begin{align*}
            \left| A \right| &= n\\
            \left| B \right| &= q\\
            \left| A \times B \right| &= n \cdot q\\
        \end{align*}

        \subsection{Proving Set Claims}

        \textit{Refer to 5.10 in textbook}

        \section{Vacuous Truth}

        \begin{definition}
            \textbf{Vacuous Truth}: something is true only because of the strange convention applied to conditionals:
            \begin{displaymath}
                F \to T \equiv T
            \end{displaymath}
        \end{definition}

        \subsection{Empty Set, Example of Vacuous Truth, }

        \begin{claim}
            The empty set $ \emptyset $ is a subset of A.
        \end{claim}
        \begin{proof}
            For $ \emptyset $ to be a subset of A, the definition of subset requires that for every object $ x $, if
            $ x $ is an element of the empty set $ \emptyset $. then $ x $ is an element of A. (\hl{By the truth table of if/then
            logic, the only case the statement evaluates to false is when the hypothesis is true and the conclusion is false. Since
            the statement's hypothesis is true, the statement will always be true, regardless of the conclusion.})
        \end{proof}

        \begin{itemize}
            \item Empty set is implicitly \textbf{a subset of any set};
            \item $ \{ \emptyset \} $ is \hl{a set containing an empty set};
            \item $ \emptyset = \{ \} $ is an empty set;
        \end{itemize}

        \section{Properties of Sets}

        \subsection{Transitivity}

        For any sets $ A $, $ B $, and $ C $,if $ A \subseteq B $ and $ B \subseteq C $,then $ A \subseteq C $.

        \subsection{Another One Involving Cartesian Products}

        For any sets $ A $, $ B $, and $ C $, if $ A \times B \subseteq A \times C $ and
        $ A \neq \emptyset $, then $ B \subseteq C $.

        \section{In Practice}

        \subsection{Three Sets Problems}

        \begin{equation}\label{eq: three set equation}
            \left| A \union B \union C \right| = \left| A \right| + \left| B \right| + \left| C \right|
            - \left| B \intersect C \right| - \left| A \intersect B \right| - \left| A \intersect C \right|
            + \left| A \intersect B \intersect C \right|
        \end{equation}

        \paragraph{Example, from Homework 3, 2018} you are in a party of $ 53 $ people.
        \begin{itemize}
            \item $ 20 $ have been to New Zealand;
            \item $ 28 $ have been to Iceland;
            \item $ 33 $ have been to Estonia;
            \item $ 3 $ have been to both New Zealand and Iceland;
            \item $ 12 $ have been to both New Zealand and Estonia;
            \item $ 1 $ one have been to all places;
        \end{itemize}

        \paragraph{Solution}
        \begin{align*}
            \left| A \right| &= 20\\
            \left| B \right| &= 28\\
            \left| C \right| &= 33\\
            \left| A \intersect B \right| &= 3\\
            \left| A \intersect C\right| &= 12\\
            \left| A \intersect B \intersect C \right| &= 1\\
            \left| A \union B \union C \right| &= 53\\
            \left| B \intersect C \right| &= x\\
        \end{align*}

        Using the equation (\ref{eq: three set equation}) on page \pageref{eq: three set equation},
        \begin{align*}
            53 &= 20 + 28 + 33 - x - 3 - 12 + 1\\
            x &= 14
        \end{align*}

        \subsection{Writing Sets Proofs}

        \begin{itemize}
            \item Tuples $ (x,y) $ in sets \hl{does not have to get its types declared explicitly}; being in the set already
            declares its set;
        \end{itemize}

        \subsubsection{Unusual Set Builder Notations}

        \begin{displaymath}
            \{ a(1,2) - (1 - b)(2,3) \mid a, b \in \mathbb{R} \}
        \end{displaymath}

        is another form of tuples, and \hl{therefore can be converted into}

        \begin{align*}
            x &= a - 2(1 - b) \\
            y &= 2a - 3 (1 - b)
        \end{align*}

        \subsubsection{Set Builders with Only One Variable}

        \begin{displaymath}
            B = \{ \left(p^{3}, p^{2} \right), p \in \mathbb{R} \}
        \end{displaymath}

        If there is another set $ A $ with two variables, and a tuple value $ (x,y) \in A $, \hl{then $ (x,y) $ can
        be written as the following to fit the set $ B $}.
        \begin{align*}
            x &= p^{3} \\
            y &= p^{2} \\
        \end{align*}

        \subsubsection{Handling Squares}

        \begin{displaymath}
            x^{2} < 4
        \end{displaymath}
        is equivalent to
        \begin{displaymath}
            -2 < x < 2
        \end{displaymath}

    \end{note}
\end{document}
