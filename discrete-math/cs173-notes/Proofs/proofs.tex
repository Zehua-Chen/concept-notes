\documentclass{note}

\usepackage{float}
\usepackage{color, colortbl}
\usepackage{longtable}
\usepackage{tabu}

\definecolor{red}{rgb}{1, 0, 0}

% For ceil and floor
\usepackage{mathtools}
\DeclarePairedDelimiter\floor{\lfloor}{\rfloor}
\DeclarePairedDelimiter\ceil{\lceil}{\rceil}

\subject{CS 173}
\date{January 22 -- 23, 107}
\id{CS17310701221}
\title{Proof}

\begin{document}
    \begin{note}{Exam 1}

        \section{Proofs}

        \begin{itemize}
            \item The reader and author of the proof should \hl{agree on what's \textbf{familiar and obvious} 
            and what's \textbf{unfamiliar}}, 
        \end{itemize}

        \subsection{Universal Proofs}

        \textbf{Example Claim}, Let $ k $ be any integer and suppose that $ k $ is odd. 
        We need to show that $ k^{2} $ is odd

        \begin{itemize}
            
            \item \textbf{Direct Proof Outline}:

            \begin{enumerate}
                \item Start with known information:
                \begin{itemize}
                    \item \hl{\textbf{Variable delcaration} or \textbf{hypothesis} of if-then statement};
                    \item \textbf{Ex.} \lq\lq $ k $ is an integer and it is odd\rq\rq;
                \end{itemize} 

                \item Move toward information needing to be proved:
                \begin{itemize}
                    \item \hl{\textbf{Conclusion} of if-then statement};
                    \item \textbf{Ex.} \lq\lq $ k^{2} $ is odd\rq\rq;
                \end{itemize}

                \item \hl{At the end of the proof, put}:
                \begin{itemize}
                    \item Box;
                    \item Dots of tirangle ($ \therefore $);
                    \item \lq\lq Q.E.D\rq\rq or \lq\lq what we needed to know\rq\rq;
                \end{itemize}
            \end{enumerate}

            \item \hl{The final version of the proof should always be of the above order};

            \item Everytime a \hl{definition is expanded}, \hl{a fresh variable should be used};
            \begin{itemize}
                \item The definition $ m $ is a perfect square can be \hl{extended into $ m = j^{2} $}
            \end{itemize}
        \end{itemize}

        \subsection{Existential Proofs}

        \paragraph{Claim}
        \textbf{Ex.} There is an integer $ k $ such that $ k^{2} = 0 $.

        \paragraph{Proof}
        Zero is an integer, so that claim is true.

        \begin{itemize}
            \item Instantiate a value of the specified type;
            \item The value \textbf{must satisfy the statement};
        \end{itemize}

        \section{Disproving}

        \begin{itemize}
            \item The disproof of a \textbf{universla statement} is a \textbf{existantial statement};
            \item The disproof of an \textbf{existantial statement} is \textbf{universal statement};
        \end{itemize}

        \subsection{Unviersal Statements}

        \paragraph{Claim} Every rational number $ q $ has a multiplicative inverse.
        \paragraph{Disproof} This claim isn’t true, because we know from high school algebra that zero has no inverse.
        \begin{itemize}
            \item To diapprove a universal claim, prove an \textbf{existantial claim} of an exception value;
        \end{itemize}

        \subsection{Existantial Statements}

        \paragraph{Claim}For every integer $ k $, $ k2 +2k+1≥0 $.
        \paragraph{Disproof} Let $ k $ be an integer. Then $ (k + 1)2 \geq 0 $ 
        because the square of any real number is non-negative. But $ (k+1)^{2} = k2 +2k+1 $. 
        So, by combining these two equations, we find that $ k2 + 2k + 1 \geq 0 $.

        \section{Proof by Cases}

        \begin{itemize}
            \item Sometimes the claim may have two cases;
            \item When there are two cases, use \textbf{proof by cases};
            \begin{itemize}
                \item It is ok to have more than two cases;
                \item It is ok for casese to overlap ($ x \geq 0 $ and $ x \leq 0 $);
                \item It is ok for cases to have variables with same names, \hl{because only
                one case is active at a time};
            \end{itemize}
        \end{itemize}

        \subsection{Example}

        \paragraph{Prove:} for every $ x, x \in \mathbb{Z} $, $ x^{2} + 2 $ is not divisible by 4

        \paragraph{Proof:} By definition of division algorithm, let $ x $ be $ x = 4q + r $, where... 
        Since $ 0 \leq r < b $, there are four cases
        \begin{enumerate}
            \item $ x = 4q $, plugin $ x $ into $ x^{2} + 2 $...
            \item $ x = 4q + 1 $, plugin $ x $ into $ x^{2} + 2 $...
            \item $ x = 4q + 2 $, plugin $ x $ into $ x^{2} + 2 $...
            \item $ x = 4q + 3 $, plugin $ x $ into $ x^{2} + 2 $...
        \end{enumerate}

        \section{Rephrasing Claims}

        \paragraph{Claim} There is no integer $ k $ such that k is odd and $ k^{2} $ is even.
        \begin{itemize}
            \item Using \textbf{logical equivalence}, such statements can be converted to statements
            that can be prooved using techniques discussed;
        \end{itemize}

        \textbf{Example}, rephrasing the above claim

        \begin{enumerate}
            \item For every integer $ k $, it is not the case that $ k $ is odd and $ k^{2} $ is even;
            \item For every integer $ k $, $ k $ is not odd or $ k^{2} $ is not even;
            \item For every integer $ k $, $ k $ is not odd or $ k^{2} $ is odd.
            \item For every integer $ k $, if $ k $ is odd then $ k^{2} $ is odd.
        \end{enumerate}

        \subsection{Proof by Contrapositive}

        \begin{itemize}
            \item Since the \hl{contrapositive of a statemnt is equivalent to itself}, contrapositive
            can be used to rephrase proofs;
            \item \hl{When negating statement, do not forget \textbf{De Morgan's Law!}};
        \end{itemize}

        \textbf{Example}, given the claim: For any integer $ a $ and $ b $, if  $ a+b \geq 15 $, then $ a \geq 8 $ or $ b \geq 8 $.
        \begin{enumerate}
            \item For any integers $ a $ and $ b $, if it’s not the case that  $ a \geq 8 $ or $ b \geq 8 $, 
            then it’s not the case that $ a + b \geq 15 $.
            \item For any integers $ a $ and $ b $, if $ a < 8 $ and $ b < 8 $, then $ a + b < 15 $.
        \end{enumerate}

        \section{Writing Proofs}

        \subsection{Variables}

        \begin{itemize}
            \item To \hl{declare a variable}, use 
            \begin{center}
                \texttt{Let \underline{identifier} be \underline{value}.}
            \end{center}

            \item To \hl{describe a variable}, use 
            \begin{center}
                \texttt{Suppose \underline{identifier} is \underline{description}}(Ex. bigger than 0)
            \end{center}
        \end{itemize}
        
    \end{note}
\end{document}
