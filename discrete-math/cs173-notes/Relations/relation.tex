\documentclass{note}
    
\usepackage{float}
\usepackage{color, colortbl}
\usepackage{longtable}
\usepackage{tabu}

\definecolor{red}{rgb}{1, 0, 0}

% For ceil and floor
\usepackage{mathtools}
\DeclarePairedDelimiter\floor{\lfloor}{\rfloor}
\DeclarePairedDelimiter\ceil{\lceil}{\rceil}

\subject{CS 173}
\date{February 5 -- 10, 107}
\id{CS17310702051}
\title{Relations}

\begin{document}
    \begin{note}{Exam 5}

        \section{Relations}

        \subsection{Relation}

        \begin{center}
            \texttt{Relation \textbf{R} on a \textbf{set A}}
        \end{center}
        \begin{itemize}
            \item $ R $ is a \hl{set of ordered pairs from the set $ A $};
        \end{itemize}
        \begin{itemize}
            \item The \hl{base set $ A $ can be \textbf{infinite}};
            \item The \hl{base set $ A $ \textbf{cannot} be \textbf{empty}};
        \end{itemize}

        \subsection{Related}

        \begin{center}
            \texttt{A relation R on set A, R contains the pair (x,y), condition...}
        \end{center}
        \begin{itemize}
            \item \textbf{Shorthand}: $ x R y $
            \item $ x \not \mathrel{R} y $ means x not related to y ({\texttt not mathrel});
        \end{itemize}

        \section{Properties of Relation}

        \subsection{Reflexives vs Irreflesives}

        \begin{description}
            \item[Reflexives] whether or not all elements are \textbf{related to themselves} $ \leq $;
            \begin{displaymath}
                \forall x \in A, x R x \to R \text{ is reflexives}
            \end{displaymath}

            \item[Irreflesives] \textbf{no element} is \textbf{related to itself} $ < $;
            \begin{displaymath}
                \forall x \in A, x \not \mathrel{R} x \to R \text{ is irreflexives}
            \end{displaymath}

            \item[Neither] \textbf{some} are \textbf{related} to themselves, \textbf{some} are \textbf{not};
        \end{description}

        \begin{itemize}
            \item \hl{The negation of \textbf{reflexives} is \textbf{NOT irreflesives}};
            \begin{itemize}
                \item The negation of $ R $ is reflexive, if $ x R x $ for all $ x \in A $ is
                \begin{center}
                    $ R $ is not reflexive, there is an $ x \in A $, $ x R x $;
                \end{center}
            \end{itemize}

            \item \textbf{reflexives} and \textbf{irreflesives} are \textbf{mutually exclusive};
        \end{itemize}

	\subsubsection{Relations that are Neither Reflexive nor Irreflexive}

	A relation is neither reflexive nor irrefelsvie is a relation where \hl{some elements are related to itself}, and \hl{some elements are 
	not related to itself}.

        \subsection{Symmetric vs Antisymmetric}\label{sec: symmetric and antisymmetric}

        \begin{description}
            \item[Symmetric] \hl{order does not matter} in a pair; $ x R y $ is the same as $ y R x $;
            \begin{displaymath}
                \forall x, y \in A, x R y \to y R x
            \end{displaymath}
            \begin{itemize}
                \item Given two objects, even though they are not linked by arrows, it is still symmetric, according to the definition
                of if
                \item If only one arrow exists between two objects, then it is not symmetric, according to the definition of if;
            \end{itemize}

            \item[Antisymmetric] two \textbf{distinct} elements are \textbf{not related} in both directions; (first definition
            is better for understanding, the second is better for writing proofs)
            \begin{displaymath}
                \forall x, y \in A, x \neq y \wedge x R y \to y \not \mathrel{R} x
            \end{displaymath}
            \begin{displaymath}
                \forall x, y \in A, x R y \wedge y R x \to x = y
            \end{displaymath}
        \end{description}

        \begin{itemize}
            \item \textbf{Distinct}: not equal; to say \hl{two numbers are \textbf{distinct} means they are not equal}.
            \item The two words \textbf{Symmetric} and \textbf{Antisymmetric} are used on \textbf{relations}, and \textbf{not values};
            \item Symmetric and Antisymmetric are \textbf{not mutually exclusive};
        \end{itemize}

	\subsubsection{Relations that are Neither Symmetric Nor Antisymmetric}

	A relation is neither symmetric nor antisymmetric if it \hl{there is on pair that does not satisfy $ a R b \to b R a $} and \hl{that does
	not satisfy $ x \neq y \wedge x R y \to y \not \mathrel{R} x $}

        \subsection{Transitive}

        A relation $ R $ on set $ A $ is transitive if
        \begin{displaymath}
            \forall a,b,c \in A, \left( a R b \wedge b R c \right) \to a R c
        \end{displaymath}

        \textbf{Not transitive}
        \begin{displaymath}
            \exists a,b,c \in A, \left( a R b \wedge b R c \right) \wedge a \not \mathrel{R} c
        \end{displaymath}

        \begin{itemize}
            \item \hl{To disprove transitive statements}, find \textbf{exceptional value} (disproof of universla quantifier
            is existential quantifier);

            \item In a relation where \hl{no elements are \textbf{related}}, the relation is \textbf{transitive,
            symmetric, antisymmetric};
            \item \hl{Vacuous truth does not apply to \textbf{reflexive} and \textbf{irreflesives}};
            \item Relations where \hl{there is only one or two elements can still \textbf{be transitive}};
        \end{itemize}

	\section{Properties of a Graph with No Arrows}

	\begin{itemize}
	    \item \textbf{Transitive}: since no elements are related, it is \hl{not possible to satisfy the \textbf{hypothesis of transitive}},
	    therefore by the definition of \textbf{transitive}, the relation is transitive;

	    \item \textbf{Symmetric, Antisymmetric}: same reason as above, the hypothesis are never satisfied;

	    \item \textbf{Irreflexive, if there is no self-pointing arrows}: since there are no arrows, a node cannot be related to oneself;

	    \item Vacuous truth does not apply to \textbf{reflexive and irreflexive} as they are \textbf{unconditional} (with no if, then)

	\end{itemize}


        \section{Types of Relations}

        \begin{itemize}
            \item \textbf{Partial Order}: a relation that is \textbf{reflexive, antisymmetric, transitive};
            \item \textbf{Linear Order / Total Order}: a \textbf{partial order} $ R $ where every pair elements are
            \textbf{comparable}; For every two elements $ x $ and $ y $, either $ x R y $ or $ y R x$
            (\hl{or both, as or is inclusive or});
            \begin{itemize}
                \item \textbf{Either or} is \textbf{inclusive or};
            \end{itemize}
            \item \textbf{Strict Partial Order}: a relation that is irreflesive, antisymmetric, and transitive;
            \item \textbf{Equivalence Relation}: a relation that is \textbf{reflexive, symmetric, and transitive}. If $ R $ is
            in an equivalence relation on a set $ A $ and $ x $ is an element of $ A $, we can define the equivalence class of $ x $
            to be the set of all elementsr elated to $ x $
            \begin{displaymath}
               [x]_{R}=  \{ y \in A \mid x R y \}
            \end{displaymath}
        \end{itemize}


        \section{Graphs}

        \subsection{Transitive Closure}
        \begin{itemize}
            \item On a transitive graph, if $ A $ has arrow leading to $ B $, which has arrow leading to $ C $, then
            \hl{there must be an arrow leading from $ A $ to $ B $}.

            \item In a graph with one element that has one arrow pointing toward itself, it is \textbf{symmetric, antisymmetric,
            transitive and reflexive};
            \begin{itemize}
                \item \textbf{Antisymmetric}: refer to the second definition in section \ref{sec: symmetric and antisymmetric} on page
                \pageref{sec: symmetric and antisymmetric}, \hl{the graph means $ a R a \to a R a $, which satisfies the first part
                of the definition of antisymmetric and since $ a =  a $, the relation is antisymmetric};
            \end{itemize}
        \end{itemize}


    \end{note}
\end{document}
