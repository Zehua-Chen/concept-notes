\documentclass{note}

\usepackage{float}
\usepackage{color, colortbl}
\usepackage{longtable}
\usepackage{tabu}

\definecolor{red}{rgb}{1, 0, 0}

% For ceil and floor
\usepackage{mathtools}
\DeclarePairedDelimiter\floor{\lfloor}{\rfloor}
\DeclarePairedDelimiter\ceil{\lceil}{\rceil}

\subject{CS 173}
\date{January 16 -- 18, 107}
\id{CS17310701161}
\title{Logic}

\begin{document}
    \begin{note}{Exam 1}

        \section{Types of Logic}

        \begin{itemize}
            \item Proportinal Logic;
            \item Predicate Logic;
        \end{itemize}

        \section{Proportional Logic}

        \begin{itemize}
            \item Proportional logic: \hl{\textbf{true} or \textbf{false}}
            \item Proportional logic does not take variables;
            \item Examples:
            \begin{itemize}
                \item Jacksons likes philosophy;
                \item $ P = N \cdot P $
                \item \textbf{Not a proportional logic}: Run away! (Command)
                \item \textbf{Not a proportional logic}: Is it raining outside? (Question)
            \end{itemize}
        \end{itemize}

        \subsection{Sets}

        \begin{itemize}
            \item $ \mathbb{Z} $: $ \{ -3, -2, -1, 0, 1, 2, 3 \} $ \textbf{integers};
            \item $ \mathbb{Z}^{+} $: $ \{1, 2, 3 \} $ \textbf{positive integers};
            \item $ \mathbb{N} $: $ \{0, 1, 2, 3 \} $ \textbf{non-negative integers};
            \item $ \mathbb{R} $: \textbf{real numbers} (all \textbf{rational numbers, include irrational numbers
            $ \sqrt{12} $});
            \item $ \mathbb{Q} $: \textbf{rational numbers} ($ \dfrac{a}{b}, b \ne 0 $);
            \item $ \mathbb{C} $: \textbf{complex numbers} ($ \sqrt{-1} $)
            \item \hl{Natural numbers in CS 173 include 0};
        \end{itemize}

        \subsection{Variable Types}
        An variable $ x $ is of type $ T $
        \begin{equation}\label{eq: var type}
            x \in T
        \end{equation}

        \subsection{Intervals}
        \begin{itemize}
            \item \textbf{Closed interval}: $ \left[a, b\right] $, \hl{include a, b};
            \item \textbf{Open interval}: $ \left(a, b\right) $, \hl{does not include a, b};
            \item \textbf{half open}: $ \left(a, b\right] $
        \end{itemize}

        \subsection{Pairs of Reals}\label{sec: pairs}

        \begin{itemize}
            \item A pair of type $ R $: $ \mathbb{R}^{2}, (\mathbb{R}, \mathbb{R}) $;
            \item A triple of type $ R $: $ \mathbb{R}^{3}, (\mathbb{R}, \mathbb{R}, \mathbb{R}) $;
        \end{itemize}

        \subsubsection{Purposes with Pairs}
        \begin{itemize}
            \item Given a set, \textbf{different purposes} intended for the sets \hl{affect the type of operations
            that can be done to a type};
            \item If a pair is an interval, then it is safe to say $ z \in \left(a, b\right) $
            \item IF a pair is a point on the graph, then addition with another point will be 
            $ \left(a + c, b + d\right) $;
        \end{itemize}

        \subsubsection{ Notation for 2D Points}
        x, y are points on the graph:
        \begin{displaymath}
            \exists (x, y) \in \mathbb{R}^{2}, x^{2} + y^{2} = 1
        \end{displaymath}
        For pairs, refer to \ref{sec: pairs} on page \pageref{sec: pairs}

        \subsection{Operators}

        \begin{itemize}
            \item Not ($ \neg $, LaTeX: neg): \hl{flip true to false}, and \hl{false to true}
            \item And ($ \wedge $, LaTeX: wedge);
            \item But, Or ($ \vee $, LaTeX: vee);
            \begin{itemize}
                \item Or in this course refers to \hl{inclusive \textbf{or}};
                \item \textbf{Inclusive Or}: \hl{either $ p $, or $ q $ or both} have to be true for the operator to be \textbf{true};
                \item \textbf{Exclusive Or}($ \oplus $): \hl{either $ p $, or $ q $ or \textbf{but not both}} have to be true for the operator to be \textbf{true};
            \end{itemize}
            \item Imply ($ p \to  q $, LaTeX: to);
            \begin{itemize}
                \item $ p $ is the \textbf{hypothesis};
                \item $ q $ is the \textbf{conclusion};
                \item \hl{Implication in CS is \textbf{timeless}}, $ q $ can happen after $ p $;
            \end{itemize}
        \end{itemize}

        \subsubsection{Truth Tables}

        \begin{itemize}
            \item There are only a limited amount of possbilities for the operands of the operators;
            \item The number of rows in the table is quadratic ($ 2^{operands} $) of the number of operands;
        \end{itemize}

        \begin{longtabu} to \textwidth{ | X[l, 1] | X[l, 1] | X[l, 2] | }
            \caption{Inclusive But, Or Operator}\\
            \hline
            $ p $ & $ q $ & $ p \vee q $ \\ \hline \hline
            T & T & T \\ \hline
            T & F & T \\ \hline
            F & T & T \\ \hline
            F & F & F \\ \hline
        \end{longtabu}

        \begin{longtabu} to \textwidth{ | X[l, 1] | X[l, 1] | X[l, 2] | }
            \caption{Exclusive But, Or Operator}\\
            \hline
            $ p $ & $ q $ & $ p \oplus q $ \\ \hline \hline
            T & T & F \\ \hline
            T & F & T \\ \hline
            F & T & T \\ \hline
            F & F & F \\ \hline
        \end{longtabu}

        \begin{longtabu} to \textwidth{ | X[l, 1] | X[l, 1] | X[l, 2] | }
            \caption{Implication Operator}\\
            \hline
            $ p $ & $ q $ & $ p \to q $ \\ \hline \hline
            T & T & T \\ \hline
            \rowcolor{red}
            T & F & F \\ \hline
            F & T & T \\ \hline
            F & F & T \\ \hline
        \end{longtabu}

        \subsection{Permutations and Logs}

        \begin{itemize}
            \item In CS, the default base of $ \log $ is $ 2 $, $ \log b \to \log_{2} b $ 
        \end{itemize}

        \subsection{Constants}
        \begin{itemize}
            \item $ T $ is a constant for \textbf{true};
            \item $ F $ is a constant for \textbf{false};
            \item $ xxx \equiv F $ means the statement $ xxx $ is \textbf{false};
        \end{itemize}

        \subsection{Converse}
        \begin{itemize}
            \item \textbf{Converse} is \hl{going backward in an \textbf{implication}};
            \item Converse \hl{does not mean equivalence, \textbf{unless specified}};
            \item \textbf{Ex}, the converse of $ p \to q $ is $ q \to p $;
        \end{itemize}

        \subsubsection{Truth Table}
        \begin{longtabu} to \textwidth{ | X[l, 1] | X[l, 1] | X[l, 2] | }
            \caption{Converse Implication Operator}\\
            \hline
            $ p $ & $ q $ & $ q \to p $ \\ \hline \hline
            T & T & T \\ \hline
            T & F & T \\ \hline
            \rowcolor{red}
            F & T & F \\ \hline
            F & F & T \\ \hline
        \end{longtabu}

        \subsection{Bidirectional}
        \begin{itemize}
            \item \textbf{Bidirectional} are \textbf{implications} \hl{taht works in two directions};
        \end{itemize}
        \begin{longtabu} to \textwidth{ | X[l, 1] | X[l, 1] | X[l, 2] | }
            \caption{Bidirectional Implication Operator}\\
            \hline
            $ p $ & $ q $ & $ q \Leftrightarrow p $ \\ \hline \hline
            T & T & T \\ \hline
            \rowcolor{red}
            T & F & F \\ \hline
            \rowcolor{red}
            F & T & F \\ \hline
            F & F & T \\ \hline
        \end{longtabu}

        \subsection{Contrapositive}
        \begin{enumerate}
            \item \textbf{Swapping} $ p $ and $ q $;
            \item \textbf{Negating} $ q $ and $ p $;
        \end{enumerate}
        \begin{itemize}
            \item Contrapositive is the \hl{same as its original content};
        \end{itemize}

        Given 
        \begin{displaymath}
            p \to q
        \end{displaymath}
        the contrapsotive is
        \begin{equation}\label{eq: contrapositive}
            \neg q \to \neg p
        \end{equation}

        \subsubsection{Truth Table}
        \begin{longtabu} to \textwidth{ | X[l, 1] | X[l, 1] | X[l, 1] | X[l, 1] | X[l, 1] | }
            \caption{Contrapositive Operator}\\
            \hline
            $ p $ & $ q $ & $ \neg q $ & $ \neg p $ & $ \neg q \Leftrightarrow \neg p $ \\ \hline \hline
            T & T & F & F & T\\ \hline
            \rowcolor{red}
            T & F & T & F & F\\ \hline
            F & T & F & T & T\\ \hline
            F & F & T & T & T\\ \hline
        \end{longtabu}

        \subsection{De Morgan's Logic / Rules of Negation}

        \begin{align*}
            \neg \left(\neg q\right) &\equiv q\\
            \neg \left( p \wedge q \right) &\equiv \neg p \vee \neg q\\
            \neg \left( p \vee q \right) &\equiv \neg p \wedge \neg q\\
            \neg \left( p \to q \right) &\equiv p \wedge \neg q\\
            p \to q &\equiv \left( \neg p \right) \vee q\\
        \end{align*}

        \subsubsection{Negating Does-Not-Exist Statements}\label{sec: negate does not exist}
        \begin{align*}
            \neg &\left( \neg \exists y, z \in A, P \left( y \right) \wedge P \left( z \right) \wedge y \neq z \right)\\
            &\exists y, z \in A, P \left( y \right) \wedge P \left( z \right) \wedge y \neq z\\
        \end{align*}

        \textit{Taken from Examlet 1, Logic Study Problems, Problem 3a}

        \subsubsection{Examples}
        \begin{description}
            \item[Ex] Negate $ a \to \left(b \to c \right) $
            \begin{align*}
                a \to \left(b \to c \right) &\equiv \neg a \vee \left(b \to c\right)\\
                &\equiv \neg a \vee \left( b \to c \right)\\
                &\equiv \neg a \vee \left( c \vee \neg b \right)\\
                &\equiv \neg a \vee \neg b \vee c\\
            \end{align*}
        \end{description}

        \subsection{Distributive Laws}
        \begin{align*}
            r \vee \left( p \wedge q \right) &\equiv \left(r \vee p\right) \wedge \left(r \vee q \right)\\
            r \wedge \left( p \vee q \right) &\equiv \left(r \wedge p\right) \vee \left(r \wedge q \right)
        \end{align*}

        \section{Predicate Logic}

        \begin{itemize}
            \item Predicate logic \hl{have variables in the logic statement};
            \item Once the variables are \hl{replaced with actual values, predicate logic become 
            \textbf{proposition logics}};
        \end{itemize}

        Given the predicate logic 
        \begin{displaymath}
            x > 2
        \end{displaymath}
            if $ x $ is 3, then
        \begin{displaymath}
            3 > 2
        \end{displaymath}
            is true.

        \subsection{Quantifiers}

        \begin{itemize}
            \item \textbf{Quantifiers} are languages that \hl{how many elements in the domain \textbf{satisfy a domain}};
            \item Types of Quantifiers
            \begin{description}
                \item[Universla Quantifiers]: \textit{for all} $ \forall $;
                \item[Existential Quantifiers]: \textit{there exists} $ \exists $;
                \item[Unique Existance]: \textit{There is a unique xxx} $ \exists ! $;
            \end{description}
        \end{itemize}

        \begin{displaymath}
            \forall x < \mathbb{Z}, x^{2} \geq 0
        \end{displaymath}

        \begin{itemize}
            \item $ \forall $: \textbf{Quantifier}
            \item $ x $: \textbf{Bound Variable}
            \item $ \mathbb{Z} $: \textbf{Domain, Replacement Set}
            \item The scope of variable $ x $ is within this statement
        \end{itemize}

        \subsection{Quantifier Negation}

        \begin{itemize}
            \item The opposite of $ \forall $ is $ \exists $;
            \item The opposite of $ \exists $ is $ \forall $;
        \end{itemize}

        \begin{description}
            \item[Ex] Negate $ \forall \text{ cookies } c, \left( \neg s \left(c\right) \wedge g \left(c\right) \right) \to \neg h \left(c\right) $
            \begin{align*}
                &\equiv \exists \text{ cookies } c, \neg \left[ \left( \neg s (c) \wedge g (c) \right) \to \neg h (c) \right]\\
                &\equiv \exists \text{ cookies } c, \left( \neg s (c) \wedge g (c) \right) \wedge \neg \neg h (c\\
                &\equiv \exists \text{ cookies } c, \neg s (c) \wedge g (c) \wedge h (c)\\
            \end{align*}
        \end{description}

        \subsection{Scope}

        \begin{itemize}
            \item If a variable is not 
            \begin{itemize}
                \item bound;
                \item assigned value;
                \item given a set of replacement values;
            \end{itemize}
            \hl{it is said to be free}

            \item Statements containing textbf{free} variabels \hl{do not have truth values},
            and therefore \hl{cannot be used in a step in proof};
        \end{itemize}

        \subsubsection{With Theorems}

        \begin{displaymath}
            \text{Theorem:} \exists x \in \mathbb{Z}, x^{2} = 0
        \end{displaymath}

        \begin{itemize}
            \item Variables stays inside theorem: \hl{x goes out of scope here!}
        \end{itemize}

        \begin{displaymath}
            \text{So, } x^{2} + 2x + 1 = 1
        \end{displaymath}

        \begin{itemize}
            \item $ x $ is \textbf{undefined};
        \end{itemize}

        \subsubsection{With Normal Statements}

        \begin{displaymath}
            \exists x \in \mathbb{Z}, x^{2} = 0 \text{ and } \exists x \in \mathbb{Z}, x^{2} = 1, 
        \end{displaymath}

        \begin{itemize}
            \item The first $ x $ syas in the scope until \textbf{and};
            \item At \textbf{and}, teh second $ x $ \textbf{masks} the first $ x $;
            \item Since the first $ x $ is masked, the statement after $ x $ is true;

            \item Another way to write the statement (\hl{Use \textbf{two variables}}):
            \begin{align*}
                \exists x \in \mathbb{Z} &, x^{2} = 0 \text{ and}\\
                \exists y \in \mathbb{Z} &, y^{2} = 1
            \end{align*}
        \end{itemize}

        % Not included in lectures

        \section{Handy Functions}

        \subsection{Logs and Exponentials}
        \begin{displaymath}
            \log_{b}x = \log_{b} a \cdot \log_{a} x
        \end{displaymath}

        \subsubsection{Factorials}
        \begin{equation}\label{eq: factorial}
            k ! = 1 \cdot 2 \cdot ... \cdot \left(k - 1\right) \cdot k
        \end{equation}

        \begin{displaymath}
            0 ! = 1
        \end{displaymath}

        \subsection{Probabilities}

        \subsubsection{Arranging}
        Given a set with $ n $ elements, there are 
        \begin{equation} \label{eq: arranging}
            n!
        \end{equation}
        (\ref{eq: factorial}) ways to arrange them into an order;

        \subsubsection{Permutation}
        \begin{itemize}
            \item Given a set of $ n $ elements, to pick $ k $ elements, there are $ y $ ways:
            \begin{equation}\label{eq: unordered}
                y = \frac{n!}{k! \left(n - k\right)!} 
            \end{equation}

            \item \ref{eq: unordered} can be abbreviated as 
            \begin{equation} \label{eq: abbrievated unordered}
                \binom{n}{k}
            \end{equation}
            \textbf{n choose k}
        \end{itemize}
        
        \subsection{Max Function}
        Max function returns the largest of its inputs
        \begin{equation}\label{eq: large func}
            max(a, b): \text{largest of a, b}
        \end{equation}

        \textbf{Example}, 
        \begin{align*}
            f(x) &= max\left( x^{2}, 7 \right)\\
            f(2) &= max\left( 4, 7 \right) = 7
        \end{align*}

        \subsection{Floor and Ceiling}
        \subsubsection{Floor}
        \begin{itemize}
            \item Floor takes \hl{\textbf{real numebers} are inputs}, and returns an \textbf{integer} that is 
            \hl{\textbf{nearest} and \textbf{no bigger} than the input};
            \begin{itemize}
                \item If the input is \hl{already integer, the input is returned};
            \end{itemize}
            \item Floor is represented as $ \floor*{x} $;
            \item Examples
            \begin{itemize}
                \item $ \floor*{3.44} = 3 $
                \item $ \floor*{3} = 3 $
                \item $ \floor*{-3.44} = -4 $
            \end{itemize}
        \end{itemize}

        \subsubsection{Ceiling}
        \begin{itemize}
            \item Ceiling takes \hl{\textbf{real numebers} are inputs}, and returns an \textbf{integer} that is 
            \hl{\textbf{nearest} and \textbf{no smaller} than the input};
            \begin{itemize}
                \item If the input is \hl{already integer, the input is returned};
            \end{itemize}
            \item Floor is represented as $ \ceil*{x} $;
            \item Examples
            \begin{itemize}
                \item $ \ceil*{3.44} = 4 $
                \item $ \ceil*{3} = 3 $
                \item $ \ceil*{-3.44} = -3 $
            \end{itemize}
        \end{itemize}

        \newpage
        \appendix
        \section{Missed Problems}

        \subsection{i}

        \begin{itemize}
            \item Try to get rid of $ i^{2} $ by subsituting $ \sqrt{- 1} $ anytime you can;
        \end{itemize}

        \subsection{Change of Log Base}

        \begin{displaymath}
            \log_{b}x = \log_{b}a \cdot \log_{a} x
        \end{displaymath}

        \subsection{Logic}

        \begin{itemize}
            \item When it comes to problems like \textit{when do the following expression evalulate to true}, \hl{the only way
            is to make a truth table};

            \item When it comes to problems like \textit{prove the following expressions are not logically equivalent}, \hl{the only way
            is to have \textbf{an exception}};

            \item To negating a $ \neg \exists x \in A, ... $ like statement, simply get rid of the $ \neg $ and get 
            \begin{displaymath}
                \exists x \in A, ...
            \end{displaymath} 
            \textit{Refer to section \ref{sec: negate does not exist} on page \pageref{sec: negate does not exist}}
        \end{itemize}

    \end{note}
\end{document}