\section{Probabilities}

  \subsection{Random Variables}

    A \textbf{random variable} is a \ul{variable whose value depends on an random
    event}; can also be considered as a function that takes an event as input and
    gives some kind of output

    \begin{itemize}
      \item \textbf{Discrete} random varaibles have values that are clearly
      separated from other variables
      \item \textbf{Continuous} random varaibles have values that are
      different, but not so different from each other. ex. $ 1.0 $ and $ 1.01 $
    \end{itemize}

  \subsection{Probabilities}

    \begin{itemize}
      \item \textbf{Variables} are typically \ul{capitalized}
      \item \textbf{Values} are typically \ul{not capitalized}
    \end{itemize}

    \begin{align}
      P\left( \text{variable} = \text{value} \right) &= x \\
      P\left( E \right) &= x
    \end{align}

    $ E $ is event; $ P $ means what percentage of time does the
    variable has the value or when the event holds true

    \begin{equation}
      P\left( X = v \text{and} Y = w \right)
    \end{equation}

    $ P $ is how often do we see $ X $ has the value of $ v $ and $ Y $ has
    the value of $ w $

    \begin{equation}
      P\left( v, w \right)
    \end{equation}

    The author hope the reader can guess what values these variables belong
    to.

  \subsection{}

  \subsection{Distribution}

    A distribution is an assignment of probability values to
    all events of interest; or all values for particular random variable or
    pair of random variables

    \subsubsection{Properties}

    $ P(A, B, C) = P(C) \cdot P(A | C) \cdot P(B | A, C) $

    \subsection{Independence}

      A, B are indenpendent iff

      $ P(A, B) = P(A) \cdot P(B) $ equivocally

      $ P(A|B) = P(A) $
      $ P(B|A) = P(B) $