\section{Overview}

  \subsection{High Level Idea}

    \begin{itemize}
      \item Train the classifier on the training data
      \item Tune any parameters (e.g. the smoothing constant)
      by trying to classify the development data
      \item Discover how well we did using the final test data
    \end{itemize}

  \subsection{Usage}

    \begin{itemize}
      \item Labeling low-level input (ex. pictures)
      \item Speech recognition
      \item Make decisions
    \end{itemize}

  \subsection{Overall System}

    \begin{tabu} to \columnwidth{ | X[1, l] | X[1, l] | X[1, l] | X[1, l] | X[1, l] | }
      \hline
      & Speech Recognizers & Object Identification & Parsing & Word Meanings \\ \hline
      Raw Input & Waveform & Image & Text & Text \\ \hline
      Low Level Features & MFCC Features & SIFT Feature Edges & Stems, Suffixes, Word bigrams & Word neighbors \\ \hline
      Low Level Labels & Phone Probabilities & Local Object ID & POS Tag & Word Embeddings \\ \hline
      Output Labels & Word Sequence & Scene Description & Parse Tree & Word Classes \\ \hline
    \end{tabu}

  \subsection{Algorithms}

    \begin{itemize}
      \item \textbf{Deterministic transformations}
      \item \textbf{Classifiers with learned parameters}
      \begin{itemize}
        \item Statistical (ex. naive bayes)
        \item Simple traditional classifiers (ex. decision tree, k-nearest neighbors)
        \item Neural
      \end{itemize}
    \end{itemize}

  \subsection{Tuning Parameters}

    Parameters of classification algorithms can be tuned to produce the
    best results:

    \begin{itemize}
      \item Values learned from the training set
      \item Tuning constants adjusted using the development data
      (ex. Laplace smoothing constant in naive Bayes)
    \end{itemize}

    Models can also be redesigned \ul{mannually}

    \begin{itemize}
      \item General design of algorithm (neural net vs HMM)
      \item Geometry of the model (ex. the number of units and
      the connections in a Bayes net or neural net)
      \item Theory-based parameters (ex. a word must have at least one vowel)
    \end{itemize}

  \subsection{Supervision}

    \begin{itemize}
      \item \textbf{Supervised} (full, explicit annotations and evaluations)
      \item \textbf{Unsupervised} (algorithm must figure out how to organize data)
      \item \textbf{Semi-supervised} (some explicit answers plus lots of raw data)
      \item \textbf{Self-supervised} (algorithms seeks out answers)
    \end{itemize}

    Feedback is always noisy, algorithms must be able to handle annotation
    mistakes.