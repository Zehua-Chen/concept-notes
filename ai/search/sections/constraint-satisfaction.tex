\section{Constraint Satisfaction}

  \begin{itemize}
    \item Result matters; path does not matter
    \item Need to find a goal state path doesn't matter
    \item Goal means passes some tests
    \item Constraints are interactions between values for different variables
  \end{itemize}

  \subsection{Classic Examples}

    NP complete examples
    \begin{itemize}
      \item Map coloring
    \end{itemize}

    We don't know if NP complete problems would require polynomial or
    exponential time. Best konwn algorithm is exponential.
    Applications can exploit domain-specific heuristics

    \subsubsection{N Queens}

      Given n x n board and n queens, place so that no queen threathens others

      \begin{itemize}
        \item \ul{Model}:
        \begin{itemize}
          \item One varaibel per column
          \item Possible values: n vertical position within the column
          \item Constraints: can't have two in the same row or diagnoal
        \end{itemize}
      \end{itemize}

    \subsubsection{Map Coloring}

      Color a map in a way that no neighboring regions have the same color

      \begin{itemize}
        \item Variables: regions
        \item Values: colors
        \item Constraints: extended boundaries $ \to $ different color
      \end{itemize}

    \subsubsection{Cryptarithmetic}

      A game consists of an equation of numbers represented as letters.
      The goal is to identify the value of each letter.

      \begin{itemize}
        \item Values: the letters
        \item Possible values: $ 1 ... 9 $
        \item Constraints: All letters are of different values, addition
        works as intended, leading digits aren't zero
      \end{itemize}

    \subsubsection{Waltz Line Labeling}

      A problem developed for comptuer vision

  \subsection{Search}

    Constraint satisfaction search can be generalized as follows:
    \begin{itemize}
      \item Start: no variable has a value
      \item Each step: assign a value to one variable
      \item End: all variables has a value
    \end{itemize}

    If there are $ n $ variables, all solutions are $ n $ steps away,
    therefore \textbf{DFS} is a good option.

    \begin{itemize}
      \item Don't need optimal path
      \item No loop: each step assigns a value and search cuts off at $ n $
    \end{itemize}

    \paragraph{Backtracking Search} DFS spends most of the time getting blocked
    and go up the search tree to try different options

    \subsection{When to Stop and Backtrack}

      \begin{itemize}
        \item \ul{Stupid method}: always go down $ n $ levels to create
        a complete solution; Checks if it meets a constraint; If not,
        retreat back up the DFS tree
        \item \ul{Smart method}: each variable keeps a list of possible values
        after each step, remove values that would violate the constraint.
        \ul{back up when any variable has no more values}
        \begin{itemize}
          \item Refer to forward checking and constraint propogation
        \end{itemize}
      \end{itemize}

    \subsubsection{Heuristics for Variable Assignment}

      Pick variabels that
      \begin{itemize}
        \item Variable with the fewest remaining values
        \item If there are ties, which variable constraints the most over
        other variables
      \end{itemize}

    \subsubsection{Choosing a Value}

      A good value to choose is a value that leaves the most options open
      for other variables

    \subsubsection{Constraint Satisfaction and Forward Checking}

      \paragraph{Forward Checking}
      At each step, forward checking only checks a variable sharing the same
      constraint as $ X $, i.e. \ul{adjacent} in constraint graph.

      \paragraph{Constraint Propogation}
      Constraint propogation works its way outward from $ X $'s neighbor
      to their neighbors, until it runs of neighbors to update.
      \begin{itemize}
        \item Reduces the amont of assignment to do
        \item Detects failure much earlier
      \end{itemize}

      Used at two points in the search

      \begin{itemize}
        \item Start of search
        \item Upgrade to forward checking or backtracking search
      \end{itemize}

    \subsubsection{AC-3}

      AC-3 is a fully working constriant propogation algorithm.

      \begin{itemize}
        \item We view constraints between variables as ordered pairs.
        $ \left( X, Y \right) $ represents a constraint flowing from $ Y $
        to $ X $. Revise $ \left( X, Y \right) $ prunes values in
        $ D\left( X \right) $ that are inconsistent with what's in
        $ D\left( Y \right) $

        \item Maintian queue of pairs that needs attention
        \item \ul{Initialize queue; two options}
        \begin{itemize}
          \item All constraints in the problem (during start)
          \item All arcs $ (X,Y) $ where $ Y $'s value was just set (during
          search)
        \end{itemize}

        \item \ul{Loop}
        \begin{itemize}
          \item Remove $ (X,Y) $ from queue. Call Revise $ (X,Y) $.
          \item If $ D(X) $ has become empty, halt returning a value
          that forces main algorithm to backtrack.
          \item If $ D(X) $ changed, push all arcs $ (C,X) $ onto the queue,
          but don't push $ (Y,X) $.
        \end{itemize}

        \item Stop when queue is empty
      \end{itemize}

    \subsubsection{Hill Climbing}

      Randomly pick a full set of variable assignments and try to tweak it
      to have the constraints satisfied. At each step, we change the value
      of one variable to reduce the number of conflicts.

      Hill climbing can be effective. At times this algorithm gets stuck in
      local optima. Solution is randomization
      \begin{itemize}
        \item Randomly pick a new starting configuration when stuck
        \item Add a random component to the movement algorithm. So it sometimes
        move to a higher cost configuration (aka. \textbf{simulated annealing})
      \end{itemize}