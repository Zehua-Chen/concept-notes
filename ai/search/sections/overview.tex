\section{Overview}

  \subsection{Graph: Problem Representation}

    \begin{itemize}
      \item Most search problems can be represented as a state graph
      \item Most state diagrams are done in graphs
      \item Some state graphs can be very large, therefore we need to be efficient
      \item AI typically only see a small portion of the whole graph at the beginning
    \end{itemize}

    \subsubsection{Graph Component}

      \paragraph{State}
      \begin{itemize}
        \item Each graph should have a \textbf{start state} and numerous
        \textbf{end states}
        \item \textbf{Information in states:}
        \begin{itemize}
          \item Cost so far
          \item Esimated cost of the remaining search (informed search only)
        \end{itemize}

        \item \textbf{Organization of states}:
        \begin{itemize}
          \item Not yet seen
          \item \textbf{Frontier} the semantics of the frontier determines
          the type of search
          \item Done
        \end{itemize}

        \item States can be built on-demand
      \end{itemize}

      \paragraph{Action}
      Actions are graph edges with costs

  \subsection{Types of Searches}

    \begin{itemize}
      \item Uninformed search
      \begin{itemize}
        \item BFS
        \item DFS
      \end{itemize}

      \item Informed search
    \end{itemize}

  \subsection{Missionaries and Cannibals Puzzle}

    \begin{itemize}
      \item States are sets of variable values
      \item Actions are legal changes to these values
      \item Edge costs often constant
    \end{itemize}



