\documentclass{note}

\usepackage{float}
\usepackage{color, colortbl}
\usepackage{longtable}
\usepackage{tabu}
\usepackage[english]{babel}

\definecolor{red}{rgb}{1, 0, 0}

% For ceil and floor
\usepackage{mathtools}
\DeclarePairedDelimiter\floor{\lfloor}{\rfloor}
\DeclarePairedDelimiter\ceil{\lceil}{\rceil}

\newtheorem{definition}{Definition}

\subject{CS 173}
\date{February 12 -- 13, 107}
\id{CS17310702111}
\title{Functions}

\begin{document}
    \begin{note}{Exam 6}

        \section{Functions}
        \begin{definition}
            Let $ A $ and $ B $ be sets, a function $ f $ from $ A $ to $ B $ is an \textbf{assignment of exactly one element
            from $ B $ to $ A $.}
        \end{definition}
        \begin{equation}\label{eq: function shorthand}
            f: A \to B
        \end{equation}

        \begin{itemize}
            \item $ A $ is the \textbf{domain} of the function;
            \item $ B $ is the \textbf{codomain} of the function;
            \item If $ x $ is in the set $ A $, then $ f(x) $ is an \textbf{image of $ x $};
            \item For $ f $ to be a function, \hl{\textbf{one input value} must only generate \textbf{one output value}};
            \begin{itemize}
                \item If \hl{a input does not have an output}, then $ f $ is \textbf{not a function};
                \item If \hl{a input has \textbf{multiple} output}, then $ f $ is \textbf{not a function};
            \end{itemize}
        \end{itemize}

        \subsection{Identity Functions}

        \begin{definition}
            Given a set $ A $, the identity functions $ id_{A} $ are functions that map every element of $ A $ to itself;
            \begin{align*}
                id_{A} &: A \to A\\
                id_{A}(x) &= x
            \end{align*}
        \end{definition}

        \subsection{Possibilities of Functions}

        \begin{definition}
            Given the function $ f $ on the sets $ A, B $, where $ \left| A \right| = n $ and $ \left| B \right| = p $, then the function
            $ f: A \to B $ has $ p^{n} $ possibilitis of construction of the function;
        \end{definition}

        \subsection{Image}

        \begin{definition}
            \textbf{Definition}: The image of function $ f: A \to B $ is the exact values produced when $ f $ is applied to every element
            of $ A $;
            \begin{equation}\label{eq: definition of image}
                f \left( A \right) = \{ f \left( x \right), x \in A \}
            \end{equation}
        \end{definition}

        \begin{itemize}
            \item if the values a function produces is $ \mathbb{Z} $, \hl{then the image is $ \mathbb{Z} $ and not $ \mathbb{R} $},
            even  though $ \mathbb{R} $ includes $ \mathbb{Z} $;
        \end{itemize}

        \begin{definition}
            \textbf{Pre-Image}: an value that maps to an element in the image;
        \end{definition}

        For a function $ f: A \to B $
        \begin{equation}
            \exists y \in B, \forall x \in A, f(x) \neq y
        \end{equation}

        \section{Properties of Functions}

        \subsection{Equality}

        For two functions to be equal:
        \begin{itemize}
            \item Two functions must assign the \hl{same output values to the same input values};
            \item \hl{Has the same \textbf{type signatures}}:
            \begin{itemize}
                \item \textbf{Type Signature}: the \textbf{type or range} of set $ A $ and $ B $ in
                \begin{displaymath}
                    f : A \to B
                \end{displaymath}
            \end{itemize}

            \item \textbf{Example}, the below two functions are not equivalent
            \begin{align*}
                f &: \mathbb{N} \to \mathbb{N}, f(x) = 2x\\
                f &: \mathbb{R} \to \mathbb{R}, f(x) = 2x\\
            \end{align*}
        \end{itemize}

        \subsection{Onto}

        \begin{definition}
            \textbf{Onto}: the function $ f: A \to B $ is \textbf{onto} if \textbf{the image of $ f $} is its \textbf{whole co-domain}.
            \begin{equation}\label{eq: definition of onto}
                \forall y \in B, \exists x \in A, f(x) = y
            \end{equation}
        \end{definition}

        \begin{itemize}
            \item A function is onto or not \hl{depends heavily on the type signature};
        \end{itemize}

        \subsubsection{Negating Onto}
        \begin{itemize}
            \item For a function to be \textbf{not onto}, there must be a value $ y $ from the codomain that no matter what value $ x $
            is picked from the domain does not give $ y $;
        \end{itemize}

        \subsection{One to One}

        \begin{definition}
            \textbf{One-to-One Functions}: functions that only has \hl{one pre-image for one image}; there are \hl{no two inputs that give
            the same output};
            \begin{align}
                \forall x, y \in A, x \neq y &\to f(x) \neq f(y)\\
                \forall x, y \in A, f(x) = f(y) &\to x = y
            \end{align}
        \end{definition}

        \subsubsection{Pigeonhole Principle}

        \begin{definition}
            Given $ n $ objects, and $ k $ labels to assign on those objects, \hl{if $ n > k $, then two objects must share the same label}.
        \end{definition}

        \subsubsection{Strictly Increasing is One-to-One}

        \begin{definition}
            \textbf{Strictly Increasing}: an increasing functions where
            \begin{equation}\label{eq: strictly increasing}
                x < y \to f(x) < f(y)
            \end{equation}
            \hl{in another word, there is no flat area on the function}.
        \end{definition}

        \begin{itemize}
            \item \hl{A \textbf{strictly increasing} function is a \textbf{one to one function}}.
            \item An \textbf{one to one function} is \textbf{not nessarily strictly increasing};
            \begin{itemize}
                \item A one to one function can also be \textbf{decreasing};
                \item Definition of one to one:
                \begin{displaymath}
                    \forall x, y \in A, x \neq y \to f(x) \neq f(y)
                \end{displaymath}
            \end{itemize}
        \end{itemize}

        \subsection{Bijection}

        \begin{definition}
            A function that is both \textbf{onto and one-to-one} is bijection.
        \end{definition}

        Given a set $ A $ and $ B $ in $ f: A \to B $
        \begin{itemize}
            \item For the function to be \textbf{onto}, $ A $ needs to have the same amount of elements as $ B $;
            \item For the function to be \textbf{one-to-one}, $ B $ needs to have the same amount of elements as $ A $;
            \item \hl{For the function to be \textbf{bijection}, the sets needs to have the same amount of elements};
        \end{itemize}

        \section{Permutation}

        There are $ y $ ways of ordering $ n $ objects, \hl{if all the objects are different}
        \begin{equation}\label{eq: ordering permutation}
            y = n!
        \end{equation}

        \subsection{Selection}

        There are $ y $ ways of selecting $ k $ objects from a $ n $ set of objects
        \begin{equation}\label{eq: select permutations}
            y = \frac{n !}{\left( n - k\right)!}
        \end{equation}

        \subsection{Duplicates}

        There are $ y $ ways of ordering $ n $ objects, where there are $ n_{1} $ of type 1, $ n_{2} $ of type 2...
        \begin{equation}\label{eq: duplicate permutation}
            y = \frac{n !}{n_{1}! n_{2}!...}
        \end{equation}

        \section{Probabilities and Functions}

        \paragraph{Functions}
        Given two sets $ A, B $,
        \begin{align*}
            \left| A \right| &= p\\
            \left| B \right| &= q\\
            p &\leq q\\
        \end{align*}
        \hl{There are $ x $ \textbf{one-to-one} functions from $ A $ to $ B $}
        \begin{displaymath}
            x = q^{p}
        \end{displaymath}

        \paragraph{One to one}
        Given two sets $ A, B $,
        \begin{align*}
            \left| A \right| &= p\\
            \left| B \right| &= q\\
            p &\leq q\\
        \end{align*}
        \hl{There are $ x $ \textbf{one-to-one} functions from $ A $ to $ B $}
        \begin{displaymath}
            x = \frac{q!}{\left(q - p\right)!}
        \end{displaymath}

        \paragraph{Onto}
        Given two sets $ A, B $,
        \begin{align*}
            \left| A \right| &= p\\
            \left| B \right| &= q\\
            p &\leq q\\
        \end{align*}
        \hl{There are $ x $ \textbf{onto} functions from $ A $ to $ B $}

        \section{Tricky Cases}

        \paragraph{GCD}
        Given a function $ f: N^{2} \to N $, where $ f $ is
        \begin{displaymath}
            f(x,y) = gcd(x,y)
        \end{displaymath}
        \hl{This is not a function}, consider $ f(0,0) = gcd(0,0) $ is undefined.

        \paragraph{Nested Quantifier 1}
        \begin{displaymath}
            \forall m, n \in \mathbb{Z}, \exists x: x = \frac{m}{n}
        \end{displaymath}
        The statement is \textbf{false}, consider what happens if $ n = 0 $;

        \section{Writing Proofs}

        \subsection{Onto}

        \begin{itemize}
            \item Declare a varriable $ n $, find a way to show that there is an $ x $
            such that $ f(x) = n $;
            \begin{enumerate}
                \item Declare the variable $ n $, but do not assign a value to $ n $(x needs to be \textbf{arbitrary});
                \item Make $ f(x) = n $;
                \item Figure out an $ x $ such that $ f(x) = n $;
            \end{enumerate}
        \end{itemize}


    \end{note}
\end{document}