\documentclass{note}

\usepackage{float}
\usepackage{color, colortbl}
\usepackage{longtable}
\usepackage{tabu}
\usepackage{tcolorbox}
\usepackage[english]{babel}
\usepackage{soul}

\definecolor{red}{rgb}{1, 0, 0}

% For ceil and floor
\usepackage{mathtools}
\DeclarePairedDelimiter\floor{\lfloor}{\rfloor}
\DeclarePairedDelimiter\ceil{\lceil}{\rceil}

\newtheorem{definition}{Definition}
\newtheorem{theorem}{Theorem}

\subject{CS 173}
\date{February 25 -- 27, 107}
\id{CS17310702191}
\title{Induction}

\begin{document}
    \begin{note}{Exam 7}

        \section{Math Review}

        \subsection{Summation}

        If $ a_{i} $ is some formula that depends on $ i $, then the summation of $ a_{1} \to a_{n} $ is
        \begin{equation}\label{eq: summation}
            \sum_{i=1}^{n} a_{i} = a_{1} + a_{2} + ... + a_{n}
        \end{equation}

        the product is
        \begin{equation}\label{eq: product}
            \prod_{i = 1}^{n} a_{i} = a_{1} \cdot a_{2} \cdot ... \cdot a_{n}
        \end{equation}

        \begin{itemize}
            \item Certain sums can be \textbf{expressed in a closed form}, \hl{where there is no summation sign}
            \item Summation in CS 173 \hl{only deals with \textbf{finite series}};
        \end{itemize}

        \subsubsection{Closed Forms}

        \begin{equation}\label{eq: closed form 1}
            \sum_{i = 1}^{n} \frac{1}{a_{i}} = 1 - \frac{1}{2^{n}}
        \end{equation}

        \begin{equation}\label{eq: geometric series}
            \sum_{k = 0}^{n} r^{k} = \frac{r^{n + 1} - 1}{r - 1}
        \end{equation}

	\subsection{Strings}

	\begin{definition}
	    \textbf{Strings} in computer science in a character sequence with \textbf{finite length}
	\end{definition}

    	\begin{itemize}
	    \item $ \epsilon $ is used to represent the \textbf{string with no characters};
	\end{itemize}

	\subsubsection{String Concatenation}

	\begin{itemize}
	    \item To \hl{concatenate two strings, simply put them together};
		\begin{itemize}
		    \item $ a = \text{Jackson} $;
		    \item $ b = \text{is a philosopher} $;
		    \item $ ab = \text{Jackson is a philosopher} $;
		\end{itemize}
	\end{itemize}

	\subsubsection{Bit Strings}

	\begin{definition}
	    A \textbf{bit string} is a string that contains only characters $ 1 $ and $ 0 $.
	\end{definition}

	\subsubsection{Sub Sets}

	\begin{definition}
	    Given a string $ A $, $ A* $ is the string that contains \textbf{characters from $ A $}.
	\end{definition}

	\begin{itemize}
	    \item $ A* $ (of $ A $ above) \hl{also contains $ \epsilon $};
	\end{itemize}

	\subsubsection{Patterns}

	\begin{itemize}
	    \item We use \textbf{regular expressions} to \hl{describe a pattern of a set of similar strings};
		\begin{itemize}
		    \item $ a \mid b $ means either the character $ a $ or the character $ b $;
		    \item $ a* $ means \textbf{zero or more copies} of the character $ a $;
		\end{itemize}
	    \item $ () $ are used to show groupings;
	    \item \textbf{Examples}:
		\begin{itemize}
		    \item $ ab* = \text{a, ab, abb ...} $;
		    \item $ c \left( b \mid a \right) * c = \text{cc, cac, cbbac ..} $;
		\end{itemize}
	\end{itemize}

	\section{Induction}

	\subsection{Induction Proof Format}

	\begin{center}
		Given a statement $ P(x) $
	\end{center}

	\begin{tcolorbox}

		\textbf{Proof}: by induction on $ x $

		\textbf{Base}: let $ x = 0 $, prove $ P(0) $

		\textbf{Inductive Hypothesis} Suppose \ul{\textbf{inductive hypothesis}} ($ P(x) $ holds for
		$ x = 1...(k - 1) $).
		\begin{itemize}
			\item Typically used to replace recursive part in \ul{Inductive Proof}
		\end{itemize}

		\textbf{Inductive Proof}: Show that $ P(x), x = k $.

		\begin{center}
			Proof of $ P(x), x = k $
		\end{center}
	\end{tcolorbox}

	\subsection{Components of Induction Proof}

	\begin{itemize}
		\item \textbf{Inductive Variable}: an \textbf{integer} variable;
	\end{itemize}

	\subsection{Styling}

	\begin{itemize}
	    \item \textbf{Mention} that you are doing induction proof;
	    \item Label:
		\begin{itemize}
		    \item \textbf{Induction Variable};
		    \item \textbf{Base};
		    \item \textbf{Inductive Steps};
		\end{itemize}

	    \item At the beginning of inductive steps, \hl{quote what $ P(k) $ is};
	\end{itemize}

	\section{Graph Coloring}

	\begin{theorem}
	    For an integer $ D $, if all nodes in a graph $ G $ have $ degrees \leq D $, then $ \chi G = D + 1 $
	\end{theorem}


    \end{note}
\end{document}
