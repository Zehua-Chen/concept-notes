\documentclass[11pt]{article}
\usepackage[margin=1in]{geometry}
\usepackage{newtxtext}
\usepackage{newtxmath}
\usepackage[scaled]{helvet}
\usepackage{bookmark}

\setlength{\parindent}{0pt}

\renewcommand{\familydefault}{\sfdefault}

% Code snippet style
\lstdefinestyle{python}{
  language=python,
  basicstyle=\footnotesize\ttfamily,
  commentstyle=\color{GoogleGreen},
  stringstyle=\color{GoogleRed},
  keywordstyle=\color{GoogleBlue}
}

% Tabu Line Separate
\tabulinesep=2pt

% Title formats
\titleformat{\paragraph}[runin]{\bfseries\color{MaterialGrey700}}{}{}{}[]
\titleformat{\subparagraph}[runin]{\color{MaterialGrey600}}{}{}{}[]

\title{History}
\author{Artificial Intelligence (CS 440)}

\begin{document}
  \maketitle
  \tableofcontents

  \section{Shakey}
    A∗ search was developed around 1968 as part of the
    Shakey project(from 1966 to 1972) at Stanford.

  \section{Blocks World}
    Blocks world is a planning domain from 1970s that involved
    having a bot rearrange blocks in certain ways using a robot hand.

  \section{Waltz Line Labelling}
    Constraint propagation was developed by David Waltz
    in the early 1970's for the Shakey project. The original
    task was line labelling (from Lana Lazebnik based on
    David Waltz's 1972 thesis).

  \section{STRIPS Planning}
    STRIPS planning, or the Stanford Research Institute Problem Solver,
    was developed in 1971. It is a simple version of classical planning,
    which is used to manage the high level goals and constraints of a
    complex planning problem, like waypoints for a
    robot doing assembly/disassembly tasks.

  \section{Boston Dynamics}
    Boston dynamics built a couple of dog like robots capable of not
    only walking around and behave like an animal, but also having
    the ability to communicate with other robots and solving puzzles together.
    Such as opening a door.

  \section{Google Self-driving Bike}
    April Fool’s prank by Google. I don’t think this will be asked on the exam
    Begins 2009
    https://waymo.com/
    https://www.youtube.com/watch?v=LSZPNwZex9s

  \section{McCulloch and Pitts}
    McCulloch and Pitts developed the first mathematical representation of
    a neuron in 1943, the MCP Neuron. This laid the groundwork for neural
    networks to be developed in the following decades. See more on Wikipedia
    and optionally skim through this article on it. There is nothing in lecture
    slides on this, so we probably won’t need to know much about it,
    but it's still interesting.

\end{document}