\chapter{Parsing Grammar}

\section{LL (Top Down) Parsing}

  \begin{definition}
    \textbf{LL}: parse from \textbf{left} to right and produces the
    \textbf{left} most derivation
  \end{definition}

  \paragraph{Implementation}
  \begin{equation*}
    A \to B a
  \end{equation*}
  \begin{itemize}
    \item Left hand side is the function being defined
    \item Terminals on right hande sides are commands to consume inputs
    \item Non terminals on right hande sides are subroutine calls
    \item For each production, make a function
    \begin{equation*}
      \left( \left[ tokens \right] \right) \to
      \left( \Tree, \left[ tokens \right] \right)
    \end{equation*}
    Return token stream to keep track of where we are in the token stream
  \end{itemize}

  \paragraph{Drawbacks}
  \begin{itemize}
    \item Grammar with left recursions causes a infinite loop
    \begin{itemize}
      \item See \ref{section: grammar/eliminate left recursion}
    \end{itemize}
    \item Grammar with common prefixes would confuse the function
    \begin{itemize}
      \item See \ref{section: grammar/eliminate common prefix}
    \end{itemize}
  \end{itemize}

\section{LR Parsing}

  \begin{definition}
    \textbf{LR}: \textbf{left} to right scan and produces a \textbf{rightmost}
    derivation; aka. \textbf{bottom up parsing}
  \end{definition}

  \begin{itemize}
    \item Uses a \textbf{push-down} automata
    \item \textbf{Actions}:
    \begin{itemize}
      \item \textbf{Shift}: consumes token from input
      \item \textbf{Reduce}: build trees from components
      \item \textbf{Goto}: jump to a different state, after reduce
      \item \textbf{Accept}: signal that we are done
    \end{itemize}
  \end{itemize}

  \subsection{Shift}

    \begin{enumerate}
      \item Consume a token from input
      \item Push the token and the current state
      \item Go to the next state
    \end{enumerate}

  \subsection{Reduce}

    \begin{enumerate}
      \item Pop tokens and states from the stack
      \item Return to the last state popped
      \item Construct a new tree from the popped tokens
    \end{enumerate}

  \subsection{Representing the Automata}

    Needs two table

    \begin{enumerate}
      \item \textbf{Action Table}: shift, reduce n, accept
      \item \textbf{Goto Table}: destination state
    \end{enumerate}

    rows are state numbers, and columns are the symbols

  \subsection{Constructing the Automata}

    \begin{enumerate}
      \item To create the start state, add the \textbf{transitive closure}
      of the start symbol
    \end{enumerate}
