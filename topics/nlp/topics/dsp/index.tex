\chapter{Digital Signal Processing}

\section{Math Basics}

  \begin{equation}
    \rms = \sqrt{\frac{\sum_{i = 1}^{N} v_{i}^{2}}{N}}
  \end{equation}

\section{Signal Basics}

  \subsection{Signals}

    \begin{itemize}
      \item \textbf{\Gls{signal}}: \glsdesc{signal}
      \begin{itemize}
        \item Some signals can be described mathmetically
        \item Some signals are too complicated to be described mathmetically
        \item Signals can be 1D (audio), 2D (image), 3D (movie)
      \end{itemize}

      \item \textbf{\Gls{continuous-signal}}: \glsdesc{continuous-signal}
      \begin{itemize}
        \item We need to be able to convert analogue signals to continuous signals
        for processing
      \end{itemize}
    \end{itemize}

    \subsubsection{Analogue vs Digital Signal}

      \begin{itemize}
        \item \textbf{Analogue Signal}
        \begin{enumerate}
          \item \textbf{Pro}: fast, well-behaved and degrades gracefully
          \item \textbf{Con}: incompatible with computers
          \item \textbf{Con}: needs special hardware
          \item \textbf{Con}: practical limits to signal processing
        \end{enumerate}

        \item \textbf{Digital Signal}
        \begin{enumerate}
          \item \textbf{Pro}: unlimited processing without specialized hardware
          \item \textbf{Pro}: can employ error correcting codes
          \item \textbf{Con}: needs care in acquisition
          \item \textbf{Con}: slow to process and potential loss of information
        \end{enumerate}
      \end{itemize}

    \subsubsection{Workflow of DSP Systems}

      \begin{enumerate}
        \item \textbf{A/D Converter}: convert analogue signal to digital signal
        \item \textbf{Disgital Signal Processor}: take in digital input and
        produce digital output
        \item \textbf{D/A Converter}: convert digital signal to analogue signal
      \end{enumerate}

    \subsubsection{Types of Operations}

      \begin{enumerate}
        \item \textbf{Time-domain}: scaling, shifting, addition
        \item \textbf{Correlation}: comparing one reference signal with others
        to determine similarity
        \item \textbf{Digital filtering}: let certain band of specific frequencies
        pass while blocking others
        \item \textbf{Modulation and demodulation}: amplitude modulation,
        frequency modulation, phase modulation
        \item \textbf{Discrete transformation}: represent a discrete-time signal
        in frequency domain
      \end{enumerate}

    \subsubsection{Three Fundamental Ideas of DSP}

      \begin{enumerate}
        \item Any signal can be represented using a finite number of discrete
        data points over time
        \item Any signal can be broken down into a stream of bits that can
        be stored, generated and manipulated by a computer
        \item Any practical signals can be represented as sums of sinusoids
        (or complex exponentials)
      \end{enumerate}

  \subsection{Properties of Signals}

    \begin{itemize}
      \item \textbf{\Gls{period}} ($ T $): \glsdesc{period}
      \begin{itemize}
        \item Speech waveforms are not periodic
      \end{itemize}

      \item \textbf{\Gls{frequency}} ($ f $): \glsdesc{frequency}
      \begin{align}
        f &= \frac{1}{T} \\
        \omega &= 2 \pi f \\
      \end{align}
      \begin{align*}
        y\left( t \right) &= A \sin\left( 2\pi f t + \phi \right) \\
        y\left( t \right) &= A \sin\left( w t + \phi \right) \\
      \end{align*}
      \begin{itemize}
        \item $ \omega $: angular frequency; (rad/s)
        \item $ f $: linear frequency; (Hz)
      \end{itemize}

      \item \textbf{\Gls{phase}} ($ \phi $): \glsdesc{phase}
      \begin{align}
        d &= \frac{360 \phi}{2 \pi} \\
        \phi &= \frac{2 \pi d}{360}
      \end{align}
      \begin{itemize}
        \item $ \phi $: phase in radians
        \item $ d $: phase in angle
      \end{itemize}
    \end{itemize}

    \subsubsection{Properties of Continuous Signal}

      \begin{equation*}
        x\left( t \right) = A \sin\left( 2\pi f t + \phi \right)
      \end{equation*}

      \begin{enumerate}
        \item For every fixed value of $ f $, $ x\left( t \right) $ is periodic
        with period being $ T = \frac{1}{f} $
        \item Continuous signals with distinct equations are themselves distinct
        \item Increasing $ f $ results in an increase in the rate of oscillation,
        as $ t $ is continuous, we can increase $ f $ without limit
      \end{enumerate}

    \subsubsection{Properties of Discrete Signal}

      \begin{equation*}
        x\left( n \right) = A \sin\left( \omega' n + \phi \right)
      \end{equation*}

      \begin{align}
        \omega' &= 2\pi f' \\
        f' &= \frac{f}{f_{s}}
      \end{align}

      $ f' $ is cycles per sample and $ f_{s} $ is the sampling frequency

      \begin{enumerate}
        \item A discrete time signal is periodic if $ f' $ is a rational number
        \begin{itemize}
          \item Fundamental period $ N $
        \end{itemize}

        \item Discrete time signals whose frequencies are separated by an
        integer multiple of $ 2 \pi $ are identical
        \item The highest rate of oscillation is when $ \omega = \pi $ or
        $ f = \frac{1}{2} $
        \item \textbf{Aliasing}: if the angular frequency of a discrete time
        signal increases from $ \pi $ to $ 2\pi $, its rate of oscillation
        will decrease
      \end{enumerate}

  \subsection{Digitalisation}

    DSP requires a signal to be \textbf{sampled} in time and \textbf{quantised}
    in amplitude.

    \begin{itemize}
      \item \textbf{\Gls{sampling}}: \glsdesc{sampling}
      \item \textbf{\Gls{quantisation}}: \glsdesc{quantisation}
    \end{itemize}

    \subsubsection{Sampling}

      \begin{equation}
        t = n \Delta_{s} = \frac{n}{f_{s}}
      \end{equation}

      \begin{itemize}
        \item $ t $: continuous time in seconds
        \item $ f_{s} $: sampling frequency
        \item $ n $: number of seconds
        \item $ \Delta_{s} $: the sampling period
      \end{itemize}

    \subsubsection{Quantisation}

      \begin{itemize}
        \item \textbf{\Gls{quantisation-level}}: \glsdesc{quantisation-level}
        \begin{itemize}
          \item The number of bits determines the number of levels; given $ n $
          bits, there will be $ 2^{n} $ quantisation levels
          \item 16 bits is typically used
        \end{itemize}

        \item \textbf{\Gls{quantisation-step}}: \glsdesc{quantisation-step}
        \item \textbf{\Gls{quantisation-error}}: \glsdesc{quantisation-error}
        \begin{equation}
          e_{q}\left( n \right) = x_{q}\left( n \right) - x\left( n \right)
        \end{equation}
        \begin{itemize}
          \item There will alwyas be \gls{quantisation-error}
          \item It is always possible to talk about the signal to noise ratio
          (SNR) of a clean digital signal
          \item Distortion, i.e. clipping, makes \gls{quantisation-error}
          difficult to handle
        \end{itemize}
      \end{itemize}

      \begin{equation}
        \rms_{e} = \sqrt{\frac{\sum_{1}{N} e_{q}^{2}\left( n \right)}{N}}
      \end{equation}

  \subsection{Nyquist Frequency/Limit}

    A signal can only be recovered from a sampled signal if the signal is
    sampled at $ f_{s} $ where

    \begin{align}
      f_{s} &> 2 f \\
      \frac{f_{s}}{2} &> f
    \end{align}

    $ \frac{f_{s}}{2} $ is called the \textbf{Nyquist frequency/limit}

    \paragraph{Example}
    To sample a 100 Hz sine wave, an $ f_{s} > 200 $ is required

    \subsubsection{Aliasing}

      \begin{enumerate}
        \item Given a $ f_{s} $ a sine wave at $ f $ is indistinguishable
        from a sine wave at frequency $ f - f_{s} $
        \item Given a $ f_{s} $ and an integer $ k $, a sine wave at
        $ f $, where $ 0 < f < \frac{f_{s}}{2} $ is indistinguishable
        from a sine wave at frequency $ f + k \times f_{s} $ after being sampled
        by $ f_{s} $
      \end{enumerate}

\section{Sound Levels}

  \subsection{Sound Intensity}

    \begin{align}
      I &= \frac{P}{A} = \frac{P}{4 \pi r^{2}} \\
      I &\propto \frac{1}{r^{2}}
    \end{align}

    \textbf{\Gls{sound-intensity}}: \glsdesc{sound-intensity}

    \begin{itemize}
      \item $ P $: power in watts (W), the rate at which which energy is
      transfered by the wave
      \item $ A $: area in $ m^{2} $
      \item \textbf{Unit}: $ \frac{W}{m^{2}} $
      \item Sound intensity is inversely proportional to square of distance
      \item Human ears can hear between $ 0.000000000001 \frac{W}{m^{2}} $ and
      $ 50 \frac{W}{m^{2}} $
    \end{itemize}

  \subsection{Sound Pressure}

    \begin{align}
      p &= \frac{F}{A} \\
      &= \frac{\sqrt{P \rho c \pi^{-1}}}{2r} \\
      p &\propto \frac{1}{r}
    \end{align}

    \textbf{\Gls{sound-pressure}}: \glsdesc{sound-pressure}

    \begin{itemize}
      \item $ F $: force; interaction between wave and ambience that cause the
      wave to accelerate
      \item $ A $: area in $ m^{2} $
    \end{itemize}

  \subsection{Decibels}

    \begin{align}
      decibels &= 10 \log_{10}\left( \frac{I}{I_{\reference}} \right) \\
      &= 10 \log_{10} \left( \frac{p}{p_{\reference}} \right)^{2} \\
      &= 20 \log_{10} \left( \frac{p}{p_{\reference}} \right) \\
      I_{\reference} &= 10^{-12} \frac{W}{m^{2}} \\
      p_{\reference} &= 2 \times 10^{-15} \frac{N}{m^{2}}
    \end{align}

    \begin{align}
      SPL &= 20 \log_{10}\left( \frac{p_{\rms}}{p_{\reference}} \right)
    \end{align}

    \subsubsection{Convert between SIL and SPL}

      \begin{equation}
        SIL\left( x \right) = 10 * \log_{10}
        \left(
          \frac
          {
            \frac
            {
              \rms\left( x \right)^{2}
            }
            {
              \frac{P_{\reference}^{2}}{I_{\reference}}
            }
          }
          {I_{\reference}}
        \right)
      \end{equation}

      Here $ x $ is a collection of data

  \subsection{Relationship Between Sound Intensity and Sound Pressure}

    \begin{align}
      I &= \frac{p^{2}}{\rho \cdot c} \\
      I &\propto p^{2}
    \end{align}

    \begin{itemize}
      \item $ \rho $: density of air, $ \frac{kg}{m^{3}} $
      \item $ c $: sound speed in air, $ \frac{m}{s} $
      \item $ \rho, c $ are functions of temperature, i.e. influenced by
      temperature
      \item \textbf{Acoustic impedance}: $ \rho \cdot c $;
      unit is $ pa \cdot \frac{s}{m} $
    \end{itemize}

  \subsection{Signal Level}

    Signal levels can be quantised using dB

    \begin{itemize}
      \item We are interested in the relative level of a signal compared to
      other signals
    \end{itemize}

    Intensity of signal can be computed as an absolute value in dB

    \begin{equation}
      I_{\dB} = 10 \log_{10} \left( A_{\rms} \right)^{2}
    \end{equation}

    $ 60 \dB $ is different from $ 60 \dB \SPL $

  \subsection{Loudness}

    SPL and frequency both contributes to loudness

\section{Systems}

  \begin{itemize}
    \item \textbf{\Gls{system}}: \glsdesc{system}
    \item A system with no memory only needs to know its input at one time
    $ t $ to produce the output of $ t $
    \item A system with memory need inputs at more than one moment in the past
    to produce its outputs
  \end{itemize}

  \subsection{Causality}

    A system is causal if output at time $ n $ only depends on input and output
    up to time $ n $

  \subsection{Linear Systems}

    \begin{equation}
      y\left( t \right) =
        \sum_{j = 0}^{M} b_{j} x\left( t - j \right)
        + \sum_{i = 1}^{N} a_{i} y\left( t - ji \right)
    \end{equation}

    The output of a linear system is the sum of scaled inputs and
    scaled previous outputs.

    \begin{itemize}
      \item $ b_{x} $ scale of inputs
      \item $ a_{x} $ scale of outputs
      \item $ i $ starts from $ 1 $ because $ y\left( t \right) $ is on the left
      of the equation
      \item A linear system may only be linear over a limited range of input;
      ex. auditory system
      \item Techniques developed for linear systems can help understand
      non-linear system; ex. filter
    \end{itemize}

    \subsubsection{Linearity}

      Linearity is \gls{homogeneity} and \gls{additivity} combined

      \begin{itemize}
        \item \textbf{\Gls{homogeneity}}: \glsdesc{homogeneity}
        \item \textbf{\Gls{additivity}}: \glsdesc{additivity}
      \end{itemize}

  \subsection{Linear Time Invariant}

    \textbf{\Gls{linear-time-invariant}}: \glsdesc{linear-time-invariant}

    \begin{itemize}
      \item LTI system is an ideal model of a subset of systems; it is an
      approximation to real situation
      \item Some parts of a non-linear system can be modeled as a LTI system,
      ex. vocal tract
      \item A time-variant system may be treated as a LTI system in a short
      period of time
    \end{itemize}

    \subsubsection{Pre-Emphasis}

      \begin{equation}
        y\left( t \right) = x\left( t \right) - \alpha x\left( t - 1\right)
      \end{equation}

      \begin{itemize}
        \item Pre-emphasis is an example of a filter
      \end{itemize}

  \subsection{Convolution}

    Calculate the output of a system, based on the input and IR of the system

  \subsection{Fourier Transform}

    \begin{itemize}
      \item Fast Fourier Transform is an implementation of Fourier Transform
    \end{itemize}
