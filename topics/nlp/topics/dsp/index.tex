\chapter{Digital Signal Processing}

\section{Basics}

  \subsection{Signals}

  \begin{itemize}
    \item \textbf{\Gls{signal}}: \glsdesc{signal}
      \begin{itemize}
        \item Some signals can be described mathmetically
        \item Some signals are too complicated to be described mathmetically
        \item Signals can be 1D (audio), 2D (image), 3D (movie)
      \end{itemize}

      \item \textbf{\Gls{continuous-signal}}: \glsdesc{continuous-signal}
      \begin{itemize}
        \item We need to be able to convert analogue signals to continuous signals
        for processing
      \end{itemize}
    \end{itemize}

    \subsubsection{Analogue vs Digital Signal}

      \begin{itemize}
        \item \textbf{Analogue Signal}
        \begin{enumerate}
          \item \textbf{Pro}: fast, well-behaved and degrades gracefully
          \item \textbf{Con}: incompatible with computers
          \item \textbf{Con}: needs special hardware
          \item \textbf{Con}: practical limits to signal processing
        \end{enumerate}

        \item \textbf{Digital Signal}
        \begin{enumerate}
          \item \textbf{Pro}: unlimited processing without specialized hardware
          \item \textbf{Pro}: can employ error correcting codes
          \item \textbf{Con}: needs care in acquisition
          \item \textbf{Con}: slow to process and potential loss of information
        \end{enumerate}
      \end{itemize}

    \subsubsection{Workflow of DSP Systems}

      \begin{enumerate}
        \item \textbf{A/D Converter}: convert analogue signal to digital signal
        \item \textbf{Disgital Signal Processor}: take in digital input and
        produce digital output
        \item \textbf{D/A Converter}: convert digital signal to analogue signal
      \end{enumerate}

    \subsubsection{Types of Operations}

      \begin{enumerate}
        \item \textbf{Time-domain}: scaling, shifting, addition
        \item \textbf{Correlation}: comparing one reference signal with others
        to determine similarity
        \item \textbf{Digital filtering}: let certain band of specific frequencies
        pass while blocking others
        \item \textbf{Modulation and demodulation}: amplitude modulation,
        frequency modulation, phase modulation
        \item \textbf{Discrete transformation}: represent a discrete-time signal
        in frequency domain
      \end{enumerate}

    \subsubsection{Three Fundemental Ideas of DSP}

      \begin{enumerate}
        \item Any signal can be represented using a finite number of discrete
        data points over time
        \item Any signal can be broken down into a stream of bits that can
        be stored, generated and manipulated by a computer
        \item Any practical signals can be represented as sums of sinusoids
        (or complex exponentials)
      \end{enumerate}

    \item \textbf{\Gls{period}}: \glsdesc{period}
    \item \textbf{\Gls{frequency}}: \glsdesc{frequency}
  \end{itemize}

  \subsection{Sampling Frequency Theory}

    Guarantees that the details lost can be reconstructed.

  \subsection{Quantization}

\section{SPL, SIL}

  \begin{equation}
    SIL\left( x \right) = 10 * \log_{10}
    \left(
      \frac
      {
        \frac
        {
          \rms\left( x \right)^{2}
        }
        {
          \frac{P_{ref}^{2}}{I_{ref}}
        }
      }
      {I_{ref}}
    \right)
  \end{equation}

  Here $ x $ is a collection of data

\section{Systems}

  \subsection{Causality}

    A system is causal if output at time $ n $ only depends on input and output
    up to time $ n $

  \subsection{Linear Systems}

    \begin{equation}
      y\left( t \right) =
        \sum_{j = 0}^{M} b_{j} x\left( t - j \right)
        + \sum_{i = 1}^{N} a_{i} y\left( t - ji \right)
    \end{equation}

    The output of a linear system is the sum of scaled inputs and
    scaled previous outputs.

    \begin{itemize}
      \item $ b_{x} $ scale of inputs
      \item $ a_{x} $ scale of outputs
      \item $ i $ starts from $ 1 $ because $ y\left( t \right) $ is on the left
      of the equation
    \end{itemize}

  \subsection{Convolution}

    Calculate the output of a system, based on the input and IR of the system

  \subsection{Fourier Transform}

    \begin{itemize}
      \item Fast Fourier Transform is an implementation of Fourier Transform
    \end{itemize}
