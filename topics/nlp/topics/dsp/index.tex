\chapter{Digital Signal Processing}

\section{Signal Basics}

  \begin{itemize}
    \item \textbf{\Gls{signal}}: \glsdesc{signal}
    \item \textbf{\Gls{period}}: \glsdesc{period}
    \item \textbf{\Gls{frequency}}: \glsdesc{frequency}
  \end{itemize}

  \subsection{Sampling Frequency Theory}

    Guarantees that the details lost can be reconstructed.

  \subsection{Quantization}

\section{SPL, SIL}

  \begin{equation}
    SIL\left( x \right) = 10 * \log_{10}
    \left(
      \frac
      {
        \frac
        {
          \rms\left( x \right)^{2}
        }
        {
          \frac{P_{ref}^{2}}{I_{ref}}
        }
      }
      {I_{ref}}
    \right)
  \end{equation}

  Here $ x $ is a collection of data

\section{Systems}

  \subsection{Causality}

    A system is causal if output at time $ n $ only depends on input and output
    up to time $ n $

  \subsection{Linear Systems}

    \begin{equation}
      y\left( t \right) =
        \sum_{j = 0}^{M} b_{j} x\left( t - j \right)
        + \sum_{i = 1}^{N} a_{i} y\left( t - ji \right)
    \end{equation}

    The output of a linear system is the sum of scaled inputs and
    scaled previous outputs.

    \begin{itemize}
      \item $ b_{x} $ scale of inputs
      \item $ a_{x} $ scale of outputs
      \item $ i $ starts from $ 1 $ because $ y\left( t \right) $ is on the left
      of the equation
    \end{itemize}

  \subsection{Convolution}

    Calculate the output of a system, based on the input and IR of the system

  \subsection{Fourier Transform}

    \begin{itemize}
      \item Fast Fourier Transform is an implementation of Fourier Transform
    \end{itemize}
