\section{Regular Expression}

Regular expressions covered are \textbf{extended regular expressions}; some
parsers may only recognize a subset of these

\begin{itemize}
  \item Case sensitive
\end{itemize}

\subsection{Disjunction}

  A disjunction of characters represents a range of possible characters.
  A disjunction is enclosed in \code{[start...end]}. In case a predefined
  sequence of characters can be considered as elements in a disjunction,
  \code{[start-end]} also works

  \begin{itemize}
    \item \code{[ab]}: maches a or b
    \item \code{[12345]} mactches from 1 to 5
    \item \code{[a-z]} matches any character from a to z
  \end{itemize}

  Disjunctions can be negated by adding \code{$ \wedge $} at the beginning:
  \code{[$ \wedge $...]}

  \begin{itemize}
    \item \code{[$ \wedge $0-9]} match any characters that are not numbers
  \end{itemize}

\subsection{Optional}

  \code{a?} the element before \code{?} can either appear \textbf{once} or
  does not exist

\subsection{Kleene Star}

  \begin{itemize}
    \item \code{a*} the element before \code{*} can appear any number of times
    (including zero)
    \item \code{a+} the element before \code{+} must appear at least once
    (aka. Kleene Plus)
  \end{itemize}

\subsection{Wild Card}

  \code{.} matches any single letter. \code{.*} matches any number of any
  letter

\subsection{Anchors}

  \begin{description}
    \item[Start of line] \code{$ \wedge $The} only matches \code{The} at the
    beginning of a line
    \item[End of line] \code{$ \$ $The} only matches \code{The} at the end of
    line
    \item[Word boundaries] \code{$ \backslash $b} matches a word boundary
    \item[Non-Boundary] \code{$ \backslash $B} matches a non-boundary
  \end{description}
