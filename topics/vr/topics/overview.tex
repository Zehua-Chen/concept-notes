\chapter{Overview}

\section{What is VR}

  \subsection{Definition of VR}

    \begin{definition}
      Inducing targeted behavior in an organism by using artificial sensory
      stimulation, while the organism has little or no awareness of the
      interference.
    \end{definition}

    \begin{itemize}
      \item \textbf{Targeted experience}: engineered experience
      \item \textbf{Organism}: the user
      \item \textbf{Artificial sensory stimulation}: regular inputs are
      replaced or enhanced by artifical inputs
      \item \textbf{Awareness}: the organism having little knoweledge of
      being in a virtual environment
      \begin{itemize}
        \item Brains form spatial neural structure during VR
      \end{itemize}
    \end{itemize}

    Given good argument, even a book can be VR

    \paragraph{Interactivity} Is the user able to influence the stimuli
    through actions?
    \begin{itemize}
      \item No: \textbf{open-loop VR}
      \item Yes: \textbf{close-loop VR}
    \end{itemize}

    \paragraph{First vs Third Person} The users experience VR in first person,
    while the developers experience VR in third person and first person.
    Developers would be biased to hoping their work would succedd and their
    bodies would adopt to its flaws (the users would not be able to
    do this as quickly)

    \paragraph{More than reality} VR can have different graphic styles and
    people are able to do things in VR that are not possible in real worlds

    \paragraph{Synthetic vs Captured} some VR experiences are artificial,
    others are done using capturing technologies. Capturing the user's
    eyes and emotion can be important. Making realistic avatars is hard
    and may make the users feel uncomfortable.

    \paragraph{Health and Safety} VR sickness happen when the experience
    overwhelms the user's senses or the stimuli is not expected (ex. movement)
    \begin{itemize}
      \item \textbf{Gorilla Arms}: the weight of the user's extended arm is
      unbearable
    \end{itemize}

  \subsection{Uses of VR}

    \begin{itemize}
      \item Video game
      \item Immersive cinema
      \item Telepresence
      \item Virtual societies
      \item Empathy (ex. swap bodies)
      \item Education
      \item Virtual prototyping
      \item Health care
      \item Augmented and mixed reality
      \item New human experiences (ex. fly)
    \end{itemize}

  \subsection{Evolution of VR}

    \begin{enumerate}
      \item Staring at rectangles (ex. cave painting)
      \item Moving pictures (ex. movies)
      \item Toward convenience and portability (ex. PCs)
      \item Video games
      \item Beyond staring at a rectangle (i.e. early VRs)
      \item VR headsets
      \item Bringing people together (ex. VR socialization)
    \end{enumerate}

% What does VR require?
% What is required for comfort in VR
