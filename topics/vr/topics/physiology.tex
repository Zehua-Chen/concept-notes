\chapter{Human Physiology and Perception}

\section{Proprioception}

  The ability to sense the relative positions of parts of our bodies and
  the amount of muscular effort being involved in moving them.

\section{Vection}

  \textbf{Vection} is the illusion of self motion

  \begin{itemize}
    \item Eyes think the body is moving
    \item Balancing organs says otherwise
  \end{itemize}

\section{Adaptation}

  The perceived effect of stimuli changes over time

  \begin{itemize}
    \item FPS players experience less vection in VR
  \end{itemize}

\section{Stevens's Power Law}

  \textbf{Stevens's power law} characterizes the \textbf{exponential} relationship
  between the increase of intensity of stimulus and the perceived magnitude
  increase in sensation created by the stimulus

\section{Just Noticeable Difference (JND)}

  This is the amount that the stimulus needs to be changed so that subjects
  would perceive it to have changed in at least 50 percent of trials.

\section{Vergence Accommodation Mismatch}

  When your brain receives mismatching cues between
  \begin{itemize}
    \item The distance of a virtual 3D object (vergence)
    \item The focusing distance (accomodation) required for the eyes to focus
    on that object
  \end{itemize}

\section{Comfort}

  \begin{itemize}
    \item Trials are essential
    \item Need consistent \textbf{90 FPS}
  \end{itemize}
