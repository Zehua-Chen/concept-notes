\section{Perception of Depth}

\begin{description}
  \item[Sensory cue, cue] a piece of information derived from sensory
  stimulation and is useful for perception
  \item[Monocular depth cue] depth cue derived from a single eye
  \item[Stereo depth cue] depth cueue derived from two eyes
  \begin{itemize}
    \item There are more monocular cues than stereo cues (enable depths from
    a single picture)
    \item Monocular dpeths cues are powerful enough to create a realistic VR
    experience
    \item Stereo cues can be expensive to render and simulate and would cause
    problems if not done perfectly
    \begin{itemize}
      \item Ex. stereo headsets can cause discomfort when viewing close up
      objects
    \end{itemize}
  \end{itemize}
\end{description}

\subsection{Monocular Depth Cues}

  \paragraph{Retina Image Size} the size of objects in the picture taken
  by the retina.
  \begin{equation}
    \frac{1}{z}
  \end{equation}
  where $ z $ is the distance

  \begin{itemize}
    \item \textbf{Size constancy scaling} a brain operation on familiar objects
    that assume that when objects come closer, the distance is changing, not
    the size
    \begin{itemize}
      \item A variant of \textbf{subjective constancy}
    \end{itemize}

    \item Objects must occur naturally so that they do not conflict with other
    depth cues
    \item When the size of objects is uncertain, distance information is used
    to estimate the size
    \item One theory suggests that the actual viewing angle is different from
    \textbf{the perceived viewing angle}; visual angle is proportional to the
    \textbf{retina image size}
    \begin{itemize}
      \item Used to explain why moons near the horizon are bigger
    \end{itemize}
  \end{itemize}

  \paragraph{Height in the Vidual Field} further objects appear smaller

  \paragraph{Accomodation} optical power can be changed through accomodation.
  young people can change as much as 10 diopters, older people can only change
  less than 1 diopters

  \paragraph{Motion Parallax} parallax means relative speed difference.
  Motion parallax refers to farther objects appear to be moving slower than
  closer objects

  \paragraph{Others}
  \begin{itemize}
    \item \textbf{Interposition}: which objects are before another
    \item \textbf{Image Blur}: distances are infered from the sharpness of focus
    \item \textbf{Atmospheric}: humidity causes farther objects to have
    lower contrast
  \end{itemize}

\subsection{Stereo Depth Cues}

  \paragraph{Horopter} human perceives a single focused image over a surface
  Eyes muscle for vergence motion provides the brain about the amount of
  convergence, thereby direct information on depth.

  Left and right eyes provide different images, called
  \textbf{binocular disparity}. Human eyes can

  \begin{itemize}
    \item Converge
    \item Lateral offset
  \end{itemize}

  enabling the images from two eyes to vary only slightly

  \paragraph{Diplopia} while focusing on objects at one depth, we get images
  of objects at other depths

\subsection{Implications for VR}

  \paragraph{Incorrect Scale Perception} three problems
  \begin{itemize}
    \item Objects in VR might resemble real world objects, but the scale is
    hard to determine
    \item User's height in VR might not match that of the real world
    \item \textbf{Interpupillary distance mismatch}
    \begin{itemize}
      \item \textbf{Real world pupils farther apart than those in VR}: VR
      worlds would be larger
      \item \textbf{Real world pupils closer than those in VR}: VR worlds would
      be smaller
    \end{itemize}
  \end{itemize}

  \paragraph{Mismatches} mismatches between real world stimulis and those in
  VR
  \begin{itemize}
    \item Vergence accomodation mismatch
    \item Headset latency causes latency in stimulis
    \item To perceive most depth cues based on motion, it is important to track
    positions; many only track orientations, making motion parallax impossible
  \end{itemize}
