\section{From Cornea to Photoceptors}

\begin{itemize}
  \item \textbf{Cornea}
  \begin{itemize}
    \item Hard, transparent surface
    \item Light enters here
  \end{itemize}

  \item \textbf{Sclera}
  \begin{itemize}
    \item Hard white layer that protects the eye
    \item Covers region not covered by cornea
  \end{itemize}

  \item \textbf{Vitreous humor}: transparent mass that allows light to pass
  through
\end{itemize}

\subsection{Photoceptors}

  Each human retina contains 120 million rods and 6 million cones.

  \begin{itemize}
    \item \textbf{Rod}: triggered by low level of light
    \begin{itemize}
      \item Only rods are triggered in low light, causing humans to lose
      ense of colors
      \item May take up to 35 minutes to adopt to low light, aka
      \textbf{scotopic vision}
    \end{itemize}

    \item \textbf{Cone}: requires more light but distinguish colors
    \begin{itemize}
      \item Active in brigher light
      \item Adoption to trichromatic mode, aka \textbf{photopic vision}
      may take up to ten minutes
    \end{itemize}

    \item Receptors density varies across regions
    \begin{itemize}
      \item \textbf{Fovea} has the highest density
      \item Center of fovea has high cone density
      \item Periphery of fovea has high rod density
      \item \textbf{Blank spot} has no receptors; result of retina wiring
      \begin{itemize}
        \item Fixed using perceptual process: filling in information using
        contextual information and rapid eye movement
      \end{itemize}
    \end{itemize}

    \item \textbf{Periphery vision}: movements are easily detected on
    periphery, but not color
    \begin{itemize}
      \item Periphery has more rods than cones
    \end{itemize}
  \end{itemize}
