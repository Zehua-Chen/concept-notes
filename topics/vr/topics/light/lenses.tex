\section{Lenses}

Patterns in VR lenses design are different from patterns that has emerged
in history

\subsection{Snell's Law}

  Describes how much rays of light bend while \textbf{entering and exiting} a
  transparent material

  \paragraph{Refrective Index}
  \begin{equation}
    n = \frac{c}{s}
  \end{equation}

  \begin{itemize}
    \item $ s $: speed of light in the material
    \item $ c $: speed of light in vacuum
    \item $ n = 2 $: light travels two times slower in the material than in
    vacuum
    \item Common refrective indexe
    \begin{itemize}
      \item \textbf{Air}: $ n = 1.000293 $
      \item \textbf{Water}: $ n = 1.33 $
      \item \textbf{Crown glass}: $ n = 1.523 $
    \end{itemize}
  \end{itemize}

  Speed of light through a medium also depends on the wave length. Therefore
  we can also write the refractive index as

  \begin{equation}
    n\left( \lambda \right)
  \end{equation}

  Which is a function of $ \lambda $

  \paragraph{Crossing Material}
  \begin{align}
    n_{1} \sin \theta_{1} &= n_{2} \sin \theta_{2} \\
    \theta_{2} &= \sin^{-1} \left( \frac{n_{1}}{n_{2}} \sin \theta_{1} \right)
  \end{align}

  \begin{itemize}
    \item If $ n_{1} < n_{2} $, $ \theta_{2} $ is closer to perpendicular than
    $ \theta_{1} $
    \item If $ n_{1} > n_{2} $, $ \theta_{2} $ is farther to perpendicular than
    $ \theta_{1} $
    \begin{itemize}
      \item Light may not penerate the surface if the incoming angle is too
      large
      \item If $ \frac{n_{1}}{n_{2}} \sin \theta_{1} > 1 $, then the light
      rays are reflected; happens to swimmers below water, where they see
      a reflection, rather than the surface above
    \end{itemize}
  \end{itemize}

  \paragraph{Prisms}
  \begin{itemize}
    \item Upward light bends downard
    \item Downward light bends upward
    \item Outcoming angle is only influenced by the incoming angle, not the
    thickness of prisms, once the reflection index is fixed
  \end{itemize}

\subsection{Lenses}

  \subsubsection{Convex Lens}

    Converges rays

    \begin{itemize}
      \item \textbf{Focal point}: the point at which parallel incoming rays
      converges
      \begin{itemize}
        \item Focal point changes as the angle of incoming parallel rays
        changes
        \item Unparallel rays can still have a focal point
        \item In the real world, different colors would give different
        focal points
      \end{itemize}

      \item \textbf{Focal Length}: distance from the center of the lens to
      the focal point
      \item \textbf{Focal Plane}: the vertical plane where the focal point is
    \end{itemize}

    \begin{equation}
      \frac{1}{s_{1}} + \frac{1}{s_{2}} = \frac{1}{f}
    \end{equation}

    \begin{itemize}
      \item The light source is at $ s_{1} $
      \begin{itemize}
        \item If $ s_{1} = \infty $, then $ s_{2} = f $
        \item If $ s_{1} = f $, then $ s_{2} = \infty $, i.e. the rays never
        converge
        \item If $ s_{1} < f $, then no real image is formed, instead
        magnification happens; the magnified image appears after the light
        source
      \end{itemize}

      \item Real image is at $ s_{2} $
      \item Focal length is $ f $
    \end{itemize}

    \paragraph{Lens Maker Equation}
    \begin{equation}
      \left( n_{2} - n_{1} \right)
      \left( \frac{1}{r_{1}} + \frac{1}{r_{2}} \right)
      = \frac{1}{f}
    \end{equation}

    \begin{itemize}
      \item $ r_{1} $: radius of curvature of front of the lens
      \item $ r_{2} $: radius of curvature of back of the lens
      \item Assume \textbf{thin lens approximation}, i.e. lens thickness is
      small relative to $ r_{1}, r_{2} $
      \item Assume $ n_{1} $, which is close enough to air
    \end{itemize}

  \subsubsection{Conave Lenses}

    Rays diverge instead of converte

    \begin{itemize}
      \item Negative focal lengths can be obtained by tracing outgoing rays
      backwards
      \item Lens maker equation can be used to calculate negative focal lens
    \end{itemize}

  \subsubsection{Diopters}

    \begin{equation}
      D = \frac{1}{f}
    \end{equation}

    \begin{itemize}
      \item Concave lens has $ D < 0 $
      \item When combining multiple lenses, we add the $ D $s of each lens.
      The resulting $ D $ would represent a single \say{lens} made up of multiple
      lenses
    \end{itemize}
