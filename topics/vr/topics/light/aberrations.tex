\section{Optic Aberrations}

VR lenses have aberrations, aka. imperfections

\subsection{Chromatic Aberration}

  Prisms is able to separate light into colors. Lenses are able to do the same
  thing, separating focused images based on colors, which is undesirable.
  \textbf{This is known as chromatic aberration}

  \begin{itemize}
    \item Different colors result in different focal points
    \item Can be compensated by chaining a series of convex and concave lenses
    of different materials
  \end{itemize}

\subsection{Sphereical Aberration}

  \textbf{Sphereical Aberration} rays farther away from the center are
  refracted farther than rays closer to the center

  \begin{itemize}
    \item \textbf{Monochromatic}: nothing to do with wave lengths
    \item Cannot be fixed by moving object, lens or the image plane
    \item The image might also focus on a curved surface, Petzval surface,
    rather than on a plane (because of sphereical shape of the lens)
  \end{itemize}

  Asphereic lens is a more complicated non-sphereical surface that

  \begin{itemize}
    \item Eliminate sphereical aberration
    \item Reduce other aberrations
  \end{itemize}

\subsection{Optical Distortion}

  \textbf{Optical Distortion}: images getting streched (pinchusion distrotion)
  or compressed (barrel distrotion) toward the
  periphery\footnote{the outer perimeter}.

  Worse case can happen with \textbf{fish-eye lens}

\subsection{Astigmatism}

  \textbf{Astigmatism}: caused by the rays being off-axis in one dimension
  and on-axis in another direction; rays focus in one dimension but not the
  other

  \begin{itemize}
    \item Impossible to bring images into focus
    \item Vertical and horizontal focal lengths appear
    \item Occur to rays that are not perpendicular to the lens
  \end{itemize}

\subsection{Comma}

  \textbf{Comma}: image magnification varies when rays are farther away
  from being perpendicular

\subsection{Flare}

  \textbf{Flare}: rays from bright light source scatter and show circular
  patterns
