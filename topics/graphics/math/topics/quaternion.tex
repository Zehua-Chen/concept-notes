\chapter{Quaternion}

Quaternions are made up of four numbers

\begin{equation}
  q = \left< a, b, c, d \right>
\end{equation}

Given a rotation by angle $ \theta $ around an axis $ a $, the quaternion $ q $
can be formed by

\begin{equation}
  q =
  \left<
    \cos\frac{\theta}{2},
    a_{x} \sin\frac{\theta}{2},
    a_{y} \sin\frac{\theta}{2},
    a_{z} \sin\frac{\theta}{2}
  \right>
\end{equation}

\begin{itemize}
  \item This can be thought as a rotation around a certain axis
  \item Combining the two variables make it easier to convert to matrices
  \item All quaternions formed this way are unit length
\end{itemize}

\section{Properties}

  \begin{itemize}
    \item For every rotation other than identity, there are two quaternions
    \begin{itemize}
      \item Directions can point in both ways
    \end{itemize}

    \item Multiplying unit quaternions give you unit quaternions
    \item Quaternion multiplications are associative
    \item Quaternion multiplications are not associative
  \end{itemize}

\section{Operations}

  \begin{itemize}
    \item \textbf{Identity}
    \begin{equation}
      q_{i} = \left< 1, 0, 0, 0 \right>
    \end{equation}

    \item \textbf{Unit Quaternion}
    \begin{equation}
      \left| q \right| = \sqrt{a^{2} + b^{2} + c^{2} + d^{2}} = 1
    \end{equation}

    \item \textbf{Multiplication}: $ q_{3} = q_{1} q_{2} $
    \begin{equation}
      \begin{split}
        a_{3} &= a_{1} a_{2} - b_{1} b_{2} - c_{1} c_{2} - d_{1} d_{2} \\
        b_{3} &= a_{1} b_{2} + a_{2} b_{1} + c_{1} d_{2} - c_{2} d_{1} \\
        c_{3} &= a_{1} c_{2} + a_{2} c_{1} + b_{2} d_{1} - b_{1} d_{2} \\
        d_{3} &= a_{1} d_{2} + a_{2} d_{1} + b_{1} c_{2} - b_{2} c_{2}
      \end{split}
    \end{equation}

    \begin{itemize}
      \item This means $ q_{1} $ rotated by $ q_{2} $
    \end{itemize}

    \item \textbf{Dot Product}
    \begin{equation}
      p \cdot q = p_{0} q_{0} + p_{1} q_{1} + p_{2} q_{2} + p_{3} q_{3}
      = \left| p \right| \left| q \right| \cos \theta
    \end{equation}

    \begin{itemize}
      \item If the two quaternions are unit quaternions, this would be the
      angle between the quaternions
    \end{itemize}

    \item \textbf{Inverse}
    \begin{equation}
      q^{-1} = \left< a, -b, -c, -d \right>
    \end{equation}

    \item \textbf{Rotating Point}: given a point $ p = \left< 0, x, y, z \right> $,
    and a quaternion $ q $, the rotated point $ p' $ is
    \begin{equation}
      p' = q p q^{-1}
    \end{equation}

    \item \textbf{Rotation Matrix}
    \begin{equation}
      \begin{bmatrix}
        2\left( a^{2} + b^{2} \right) - 1 & 2\left( bc - ad \right) & 2\left( bd + ac \right) \\
        2\left( bc + ad \right) & 2\left( a^{2} + c^{2} \right) - 1 & 2\left( cd - ab \right) \\
        2\left( bd - ac \right) & 2\left( cd + ab \right) & 2\left( a^{2} + d^{2} \right) - 1
      \end{bmatrix}
    \end{equation}

    \item \textbf{Quaternion to Angle Axis}
    \begin{align}
      \theta &= 2 \cos^{-1} a \\
      v &= \frac{1}{\sqrt{1 - a^{2}}} \left( b, c, d \right)
    \end{align}

    \item \textbf{Linear Interpolation}: $ a, b $ are quaternions, $ t $ is time
    \begin{equation}
      \begin{split}
        \theta &= \cos^{-1}\left( a, b \right) \\
        \text{slerp} &= \frac{\sin\left( \left( 1 - t \right) \theta \right)}{\sin\theta} a
        + \frac{\sin\left( t \theta \right)}{\sin\theta} b
      \end{split}
    \end{equation}

  \end{itemize}

\section{Gimbal Lock}

  Quaternions are useful because it solves \textbf{gimbal lock}, aka.
  \textbf{kinematic singularity}: one or more degrees of freedom are lost when
  a specific orientation is achieved.

  \begin{itemize}
    \item Ex. when the user looks straight up, the camera would spin
    uncontrollably
  \end{itemize}


\section{Game Engines}

  \subsection{Unity}

    \begin{itemize}
      \item Unity implements quaternion as follows
      \begin{equation}
        q =
        \left<
          a_{x} \sin\frac{\theta}{2},
          a_{y} \sin\frac{\theta}{2},
          a_{z} \sin\frac{\theta}{2},
          \cos\frac{\theta}{2}
        \right>
      \end{equation}
    \end{itemize}
