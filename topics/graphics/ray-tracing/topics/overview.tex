\chapter{Overview}

\newglossaryentry{ray}
{
  name=ray,
  description={an origin and a direction}
}

\newglossaryentry{ray-casting}
{
  name=ray casting,
  description={if an ray intersects with an object}
}

\newacronym{bvh}{BVH}{Bounding Volume Hierarchy}

\begin{itemize}
  \item \textbf{\Gls{ray}}: \Glsdesc{ray}
  \item \textbf{\Gls{ray-casting}}: \Glsdesc{ray-casting}
  \item Camera shoots rays into the screen.
  \item In naive ray tracing, if ray hits light, then a pixel is lit.
  \item In more complex ray tracing, multiple rays, each is slightly different
  from others, are shot. The result of all the rays lead to the values of one
  pixel. This can also be used to perform anti-aliasing
  \item Using \acrfull{bvh} leads to $ O\left( \log n \right) $ runtime
\end{itemize}

\section{Rays}

  \begin{equation}
    P\left( t \right) = O + td
  \end{equation}

  $ P\left( t \right) $ is a point on the ray; $ t $ is the time; $ O $ is
  the origin, and $ d $ is the direction

  \begin{itemize}
    \item Rays travel in straight lines
    \item Rays do not interfere with other rays
    \item Rays travel from light source to eyes; \textbf{physics is the same under
    path reversal (reciprocity)}
  \end{itemize}

  \subsection{Types of Rays}

    \begin{enumerate}
      \item Primary rays: rays shot from pixels
      \item Secondary rays
      \item Shadow rays
      \item Light rays
    \end{enumerate}

  \subsection{Screen Space, World Space Pixel Coordinates}

\section{Basic Ray Tracing}

  \begin{algorithm}[H]
    \caption{Basic Ray Tracing}
    define objects\;
    define materials for objects\;
    define light sources\;
    define window\;
    \ForEach{pixel}{
      construct ray through the pixel\;
      compute the intersection of the ray with each object\;
      find the closest intersection\;
      \If{there is an intersection} {
        use lights and materials to compute pixel color\;
      }
      \Else{
        pixel is set to background color\;
      }
    }
  \end{algorithm}

\section{Shooting Rays}

  \subsection{Screen Space to World Space Pixel Coordinates}

    \begin{align}
      x_{w} &= s\left( c - \frac{h_{res}}{2} + 0.5 \right) \\
      y_{w} &= s\left( r - \frac{v_{res}}{2} + 0.5 \right)
    \end{align}

    \begin{itemize}
      \item $ c, r $: screen space coordinates of pixels; $ c $ is the column,
      $ r $ is the row
      \item $ s $: pixel extent
      \item $ x_{w}, y_{w} $: coordinate in world space of the pixel center
      \item $ z_{w} $ is typically $ 0 $
    \end{itemize}

  \subsection{Orthographic}

    \begin{align}
      O &= \left( x_{w}, y_{w}, z_{w} \right) \\
      d &= \left( 0, 0, -1 \right)
    \end{align}

    Rays have the same direction as negative-z and use their world space
    coordinates as origins

  \subsection{Perspective}

    \begin{align}
      e &= \left( e_{w}, e_{w}, e_{w} \right) \\
      O &= e  \\
      d &= \left( x_{w}, y_{w}, z_{w} \right) - e
    \end{align}

\section{Intersection}

  Given an implicit surface equation $ f\left( x, y, z \right) = 0 $, we
  want to find a point $ p $ that is on the surface,
  i.e. $ f\left( p \right) = 0 $. Since $ p = O + dt $, we are looking for
  $ t $ in $ f\left( O + td \right) = 0 $.

  If $ t < 0 $, then intersection is behind the screen

  \textbf{Robustly}: a solution is resistant to error and produces similar
  result for small changes in input

  \subsection{Plane}

    \begin{align}
      Ax + By + Cz + D &= 0 \\
      \left( p - a \right) \cdot n = 0
    \end{align}

    $ a $ is a \textbf{point in space} and $ n $ is the \textbf{normal vector}
    of the plane with its origin at $ a $

    Given a ray $ O + td $ we can solve for a $ t $, which gives us the
    point of intersection.

    \begin{align*}
      \left( p - a \right) \cdot n &= 0 \\
      \left( \left( O + td \right) - a \right) \cdot n &= 0 \\
      \frac{\left( \left( a - O \right) \cdot n \right)}{d \cdot n} &= t
    \end{align*}

    If the ray and the plane is parallel, then $ d \cdot n = 0 $

  \subsection{Sphere}

    Spheres are defined by

    \begin{equation}
      \left( P - G \right) \cdot \left( P - G \right) = r^{2}
    \end{equation}

    $ P $ is a \textbf{point is space}, $ G $ is the \textbf{center of sphere},
    and $ r $ is the \textbf{radius of sphere}

    Replace $ P $ with $ O + td $ and let $ f = O - G  $

    \begin{align}
      \left( d \cdot d \right) t^{2}
        + 2 \left( f \cdot d \right) t
        + f \cdot f - r^{2}
      &= 0 \\
      a t^{2} + bt + c &= \\
      \frac{-b \pm \sqrt{b^{2} - 4ac}}{2a} &= t_{0, 1} \\
      \frac{P_{0} - G}{r} &= n
    \end{align}

    \begin{itemize}
      \item $ t_{0, 1} $: rays will have two intersection points
      \item $ b^{2} - 4ac < 0 $: ray does not intersect with sphere
      \item $ b^{2} - 4ac > 0 $: ray intersects with sphere
      \item $ b^{2} - 4ac = 0 $: ray touches sphere on the surface
    \end{itemize}

    \subsubsection{Diminished Significance}

      \paragraph{Problem}
      \begin{itemize}
        \item 32 bit float point can result in error; caused by
        \textbf{diminished} significance
        \begin{itemize}
          \item Ex. small spheres that are very far away
        \end{itemize}

        $ c = f \cdot f - r^{2}  $ is problematic, if $ f $ is big and $ r $ is
        small, diminished significance would occur. If a sphere is more than
        \begin{equation}
          2^{12} r = 4096 r
        \end{equation}
        away from the ray origin, then radius has no impact on intersection
        solution.
      \end{itemize}

      \paragraph{Solution} \textbf{Hearn Baker Method}
      \begin{align}
        b^{2} - 4ac &= 4a \left( \frac{b^{2}}{4a} - c \right) \\
        &= 4d^{2} \left( r^{2} - \left( f - \left( f \cdot \hat{d} \right) \hat{d} \right)^{2} \right)
      \end{align}

    \subsubsection{Catastrophic Cancellation}

      \paragraph{Problem}
      \begin{itemize}
        \item Occurs when adding nearly equal numbers with opposite signs.
        In sphere intersection, it occurs in
        \begin{equation*}
          -b \pm \sqrt{b^{2} - 4ac}
        \end{equation*}

        when

        \begin{equation*}
          b \approx \sqrt{b^{2} - 4ac}
        \end{equation*}

        \item Many significant bits eliminate each other
        \item Few meaningful bits remain
      \end{itemize}

      We have approximations

      \begin{align}
        \tilde{x} &= x \left( 1 + \delta_{x} \right) \\
        \tilde{y} &= y \left( 1 + \delta_{y} \right)
      \end{align}

      Small relative errors

      \begin{align}
        \left| \delta_{x} \right| &= \frac{\left| x - \tilde{x} \right|}{\left| x \right|} \\
        \left| \delta_{y} \right| &= \frac{\left| y - \tilde{y} \right|}{\left| y \right|}
      \end{align}

      Relative error of the difference can be arbitarrily large

      \begin{equation}
        \left| \frac{x \delta_{x} - y \delta_{y}}{x - y} \right|
      \end{equation}

      \paragraph{Solution} Catastrophic cancellation only happens to one of the
      two quadratic solutions. Can be fixed using $ t_{0} t_{1} = \frac{c}{a} $

      \begin{equation}
        \begin{cases}
          t_{0} = \frac{c}{q} \\
          t_{1} = \frac{q}{a}
        \end{cases}
      \end{equation}

      where $ q = \frac{1}{2}\left( b + \sign\left( b \right) \sqrt{b^{2} - 4ac} \right) $
      $ \sign $ function returns 1 if $ b > 0 $ and $ - 1 $ otherwise
