\chapter{Overview}

\newglossaryentry{ray}
{
  name=ray,
  description={an origin and a direction}
}

\newglossaryentry{ray-casting}
{
  name=ray casting,
  description={if an ray intersects with an object}
}

\newacronym{bvh}{BVH}{Bounding Volume Hierarchy}

\begin{itemize}
  \item \textbf{\Gls{ray}}: \Glsdesc{ray}
  \item \textbf{\Gls{ray-casting}}: \Glsdesc{ray-casting}
  \item Camera shoots rays into the screen.
  \item In naive ray tracing, if ray hits light, then a pixel is lit.
  \item In more complex ray tracing, multiple rays, each is slightly different
  from others, are shot. The result of all the rays lead to the values of one
  pixel.
  \item Using \acrfull{bvh} leads to $ O\left( \log n \right) $ runtime
\end{itemize}
