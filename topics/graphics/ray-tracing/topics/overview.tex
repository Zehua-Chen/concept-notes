\chapter{Overview}

\newglossaryentry{ray}
{
  name=ray,
  description={an origin and a direction}
}

\newglossaryentry{ray-casting}
{
  name=ray casting,
  description={if an ray intersects with an object}
}

\newacronym{bvh}{BVH}{Bounding Volume Hierarchy}

\begin{itemize}
  \item \textbf{\Gls{ray}}: \Glsdesc{ray}
  \item \textbf{\Gls{ray-casting}}: \Glsdesc{ray-casting}
  \item Camera shoots rays into the screen.
  \item In naive ray tracing, if ray hits light, then a pixel is lit.
  \item In more complex ray tracing, multiple rays, each is slightly different
  from others, are shot. The result of all the rays lead to the values of one
  pixel. This can also be used to perform anti-aliasing
  \item Using \acrfull{bvh} leads to $ O\left( \log n \right) $ runtime
\end{itemize}

\section{Rays}

  \begin{equation}
    P\left( t \right) = O + td
  \end{equation}

  $ P\left( t \right) $ is a point on the ray; $ t $ is the time; $ O $ is
  the origin, and $ d $ is the direction

  \begin{itemize}
    \item Rays travel in straight lines
    \item Rays do not interfere with other rays
    \item Rays travel from light source to eyes; \textbf{physics is the same under
    path reversal (reciprocity)}
  \end{itemize}

  \subsection{Types of Rays}

    \begin{enumerate}
      \item Primary rays: rays shot from pixels
      \item Secondary rays
      \item Shadow rays
      \item Light rays
    \end{enumerate}

  \subsection{Screen Space, World Space Pixel Coordinates}

\section{Basic Ray Tracing}

  \begin{algorithm}[H]
    \caption{Basic Ray Tracing}
    define objects\;
    define materials for objects\;
    define light sources\;
    define window\;
    \ForEach{pixel}{
      construct ray through the pixel\;
      compute the intersection of the ray with each object\;
      find the closest intersection\;
      \If{there is an intersection} {
        use lights and materials to compute pixel color\;
      }
      \Else{
        pixel is set to background color\;
      }
    }
  \end{algorithm}

\section{Shooting Rays}

  \subsection{Screen Space to World Space Pixel Coordinates}

    \begin{align}
      x_{w} &= s\left( c - \frac{h_{res}}{2} + 0.5 \right) \\
      y_{w} &= s\left( r - \frac{v_{res}}{2} + 0.5 \right)
    \end{align}

    \begin{itemize}
      \item $ c, r $: screen space coordinates of pixels; $ c $ is the column,
      $ r $ is the row
      \item $ s $: pixel extent
      \item $ x_{w}, y_{w} $: coordinate in world space of the pixel center
      \item $ z_{w} $ is typically $ 0 $
    \end{itemize}

  \subsection{Orthographic}

    \begin{align}
      O &= \left( x_{w}, y_{w}, z_{w} \right) \\
      d &= \left( 0, 0, -1 \right)
    \end{align}

    Rays have the same direction as negative-z and use their world space
    coordinates as origins

  \subsection{Perspective}

    \begin{align}
      e &= \left( e_{w}, e_{w}, e_{w} \right) \\
      O &= e  \\
      d &= \left( x_{w}, y_{w}, z_{w} \right) - e
    \end{align}
