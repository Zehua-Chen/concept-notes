\chapter{Camera}

\section{Camera Setup}

  A virtual camera has the following properties

  \begin{enumerate}
    \item Eye point, i.e. \textbf{positions}, $ e $
    \item Look at point $ l $
    \item Up vector $ up $
    \item View plane distance $ d $
    \item View direction: $ l - e $
    \item This system will break when view direction and up vector are
    \textbf{paralle} or \textbf{antiparallel}, i.e. the angle between them is
    $ 0\deg $ or $ 180\deg $, i.e. a dot product of $ 0 $
  \end{enumerate}

\section{Orthographic}

  \begin{align}
    O &= \left( x_{w}, y_{w}, z_{w} \right) \\
    d &= \left( 0, 0, -1 \right)
  \end{align}

  Rays have the same direction as negative-z and use their world space
  coordinates as origins

  \subsection{Properties}

    \begin{itemize}
      \item All rays are parallel
      \item It is possible to zoom
      \item \Gls{foreshortening} can occur
      \item Angles are \textbf{not preserved}
    \end{itemize}

\section{Perspective}

  Perspective rays use $ e $ as $ O $, but we need $ d $:

  \begin{enumerate}
    \item Find $ d_{v} $, direction in view space
    \item Translate $ d_{v} $ into $ w_{w} $, direction in world space
  \end{enumerate}

  \subsection{View Space Direction}

    \begin{align}
      d_{v} &= \left( x_{v}, y_{v}, z_{v} \right) - e \\
      d_{v} &= \left( x_{v}, y_{v}, z_{v} \right) - \left( 0, 0, 0 \right) \\
      d_{v} &= \left( x_{v}, y_{v}, z_{v} \right) \\
      d_{v} &=
      \left(
        s\left( c - \frac{h_{res}}{2} + \delta_{x} \right),
        s\left( r - \frac{v_{res}}{2} + \delta_{y} \right),
        -d
      \right)
    \end{align}

    \begin{itemize}
      \item $ c, r $: screen space coordinates of pixels; $ c $ is the column,
      $ r $ is the row
      \begin{itemize}
        \item $ \left( 0, 0 \right) $ is at the bottom left of an image
      \end{itemize}

      \item $ \delta_{x}, \delta_{y} $: subpixel offset from sampling
      \item $ s $: pixel extent; pixel length in world space
      \item $ x_{v}, y_{v}, z_{v} $: direction of ray in view space, where
      $ e = \left( 0, 0, 0 \right) $
    \end{itemize}

  \subsection{World Space Direction}

    To convert $ d_{v} $ to $ d_{w} $, we first need to construct a world
    frame.

    \begin{align}
      w &= \frac{\left( e - l \right)}{\left\Vert e - l \right\Vert} \\
      u &= \frac{\left( up \times w \right)}{\left\Vert up \times w \right\Vert} \\
      v &= w \times u
    \end{align}

    \begin{itemize}
      \item $ u $ is parallel to pixel row;
      \item $ v $ is parallel to pixel columns;
    \end{itemize}

    From which we can convert a screen-space direction to a world-space direction

    \begin{equation}
      d_{w} = u x_{v} + v y_{v} + w z_{v}
    \end{equation}

  \subsection{Properties}

  \begin{itemize}
    \item Distortion is a feature not a bug. To reduce distortion, move
    the eye point away from objects
    \item All lines of projection meet at a single point
    \item Geometry farther away from the central axis of projection will be
    more distorted, even if it is very close to the viewplane.
    \item \Gls{foreshortening} can occur
    \item Increasing distance between the eye and the viewplane will zoom in
    \item Angles are \textbf{not preserved}
  \end{itemize}
