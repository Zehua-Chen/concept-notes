\chapter{Intersection}

Given an implicit surface equation $ f\left( x, y, z \right) = 0 $, we
want to find a point $ p $ that is on the surface,
i.e. $ f\left( p \right) = 0 $. Since $ p = O + dt $, we are looking for
$ t $ in $ f\left( O + td \right) = 0 $.

If $ t < 0 $, then intersection is behind the screen

\textbf{Robustly}: a solution is resistant to error and produces similar
result for small changes in input

\section{Plane}

  \begin{align}
    Ax + By + Cz + D &= 0 \\
    \left( p - a \right) \cdot n = 0
  \end{align}

  $ a $ is a \textbf{point in space} and $ n $ is the \textbf{normal vector}
  of the plane with its origin at $ a $

  Given a ray $ O + td $ we can solve for a $ t $, which gives us the
  point of intersection.

  \begin{align*}
    \left( p - a \right) \cdot n &= 0 \\
    \left( \left( O + td \right) - a \right) \cdot n &= 0 \\
    \frac{\left( \left( a - O \right) \cdot n \right)}{d \cdot n} &= t
  \end{align*}

  \begin{itemize}
    \item If the ray and the plane is parallel, then $ d \cdot n = 0 $
    \item If the intersection is behind the origin then $ t < 0 $
  \end{itemize}

\section{Sphere}

  Spheres are defined by

  \begin{equation}
    \left( P - G \right) \cdot \left( P - G \right) = r^{2}
  \end{equation}

  $ P $ is a \textbf{point is space}, $ G $ is the \textbf{center of sphere},
  and $ r $ is the \textbf{radius of sphere}

  Replace $ P $ with $ O + td $ and let $ f = O - G  $

  \begin{align}
    \left( d \cdot d \right) t^{2}
      + 2 \left( f \cdot d \right) t
      + f \cdot f - r^{2}
    &= 0 \\
    a t^{2} + bt + c &= \\
    \frac{-b \pm \sqrt{b^{2} - 4ac}}{2a} &= t_{0, 1} \\
    \frac{P_{0} - G}{r} &= n
  \end{align}

  \begin{itemize}
    \item $ t_{0, 1} $: rays will have two intersection points
    \item $ b^{2} - 4ac < 0 $: ray does not intersect with sphere
    \item $ b^{2} - 4ac > 0 $: ray intersects with sphere
    \item $ b^{2} - 4ac = 0 $: ray touches sphere on the surface
  \end{itemize}

  \subsection{Diminished Significance}

    \paragraph{Problem}
    \begin{itemize}
      \item 32 bit float point can result in error; caused by
      \textbf{diminished} significance
      \begin{itemize}
        \item Ex. small spheres that are very far away
      \end{itemize}

      \item $ c = f \cdot f - r^{2}  $ is problematic, if $ f $ is big and $ r $ is
      small, diminished significance would occur. If a sphere is more than
      \begin{equation}
        2^{12} r = 4096 r
      \end{equation}
      away from the ray origin, then radius has no impact on intersection
      solution.
    \end{itemize}

    \paragraph{Solution} \textbf{Hearn Baker Method}
    \begin{align}
      b^{2} - 4ac &= 4a \left( \frac{b^{2}}{4a} - c \right) \\
      &= 4d^{2} \left( r^{2} - \left( f - \left( f \cdot \hat{d} \right) \hat{d} \right)^{2} \right)
    \end{align}

  \subsection{Catastrophic Cancellation}

    \paragraph{Problem}
    \begin{itemize}
      \item Occurs when adding nearly equal numbers with opposite signs.
      In sphere intersection, it occurs in
      \begin{equation*}
        -b \pm \sqrt{b^{2} - 4ac}
      \end{equation*}

      when

      \begin{equation*}
        b \approx \sqrt{b^{2} - 4ac}
      \end{equation*}

      \item Many significant bits eliminate each other
      \item Few meaningful bits remain
    \end{itemize}

    We have approximations

    \begin{align}
      \tilde{x} &= x \left( 1 + \delta_{x} \right) \\
      \tilde{y} &= y \left( 1 + \delta_{y} \right)
    \end{align}

    Small relative errors

    \begin{align}
      \left| \delta_{x} \right| &= \frac{\left| x - \tilde{x} \right|}{\left| x \right|} \\
      \left| \delta_{y} \right| &= \frac{\left| y - \tilde{y} \right|}{\left| y \right|}
    \end{align}

    Relative error of the difference can be arbitarrily large

    \begin{equation}
      \left| \frac{x \delta_{x} - y \delta_{y}}{x - y} \right|
    \end{equation}

    \paragraph{Solution} Catastrophic cancellation only happens to one of the
    two quadratic solutions. Can be fixed using $ t_{0} t_{1} = \frac{c}{a} $

    \begin{equation}
      \begin{cases}
        t_{0} = \frac{c}{q} \\
        t_{1} = \frac{q}{a}
      \end{cases}
    \end{equation}

    where $ q = \frac{1}{2}\left( b + \sign\left( b \right) \sqrt{b^{2} - 4ac} \right) $
    $ \sign $ function returns 1 if $ b > 0 $ and $ - 1 $ otherwise

\section{Parabolic Cylinder}

  Parabolic cylinder is defined by

  \begin{equation}
    z = x^{2}
  \end{equation}

  Using

  \begin{align*}
    p_{x} &= O_{x} + t d_{x} \\
    p_{z} &= O_{z} + t d_{z}
  \end{align*}

  Parabolic cylinder can be redefined as

  \begin{align*}
    O_{z} + t d_{z} &= O_{x}^{2} + 2 O_{x} t d_{x} + t^{2} d_{x}^{2} \\
    0 &= d_{x}^{2} t^{2} + t \left( 2 O_{x} d_{x} - d_{z} \right)
      + \left( O_{x}^{2} - O_{z} \right) \\
      &= a t^{2} + bt + c
  \end{align*}

  Which is a quadratic equation
