\chapter{Shadow}

\section{Types of Lights}

  \begin{itemize}
    \item \textbf{\Gls{point-light}}: \glsdesc{point-light}
    \item \textbf{\Gls{directional-light}}: \glsdesc{directional-light}
    \item Reasons why shadows are not completely dark
    \begin{itemize}
      \item There are more than one light in a scene
      \item The shadow is shaded using phong reflection model, which has a
      ambient term
    \end{itemize}
  \end{itemize}

\section{Types of Shadows}

  \begin{itemize}
    \item Idealized lights have \textbf{hard-edge} shadows
    \item Real lgihts cast \textbf{soft} shadows
    \begin{itemize}
      \item \textbf{\Gls{umbra}}: \glsdesc{umbra}
      \item \textbf{\Gls{penumbra}}: \glsdesc{penumbra}
    \end{itemize}
  \end{itemize}

\section{Implementation}

  \begin{itemize}
    \item Determine visibility of light using rays
    \item Primary ray generates a shadow ray upon hitting an surface
    \begin{itemize}
      \item Shadow ray's origin is the hitpoint
      \item Shadow ray's direction points to the light source
      \begin{itemize}
        \item Point light: light position - hit point
      \end{itemize}
    \end{itemize}

    \item Shadow rays that hit lights should have $ t = 1 $; shadow light
    that hit other objects behind light should have $ t > 1 $
    \item Due to numerical error, primary ray's hit point might be below a
    a surface. In this case, the shadow ray will hit the surface rather than
    light, causing the images to appear to be covered by \textbf{black-pepper}.
    \begin{itemize}
      \item Solution: add an $ \epsilon $ that move the origin of shadow ray
      up a bit along its direction. Start with $ \epsilon = 0.00001 $
      \item Solution: ignore the surface from which the shadow ray is comming
      from; \textbf{fails with concave shapes}, where shadow rays needs to
      intersect the origin object
    \end{itemize}

    \item Shadows can be expensive; have to shoot a shadow ray for each light
    light
  \end{itemize}
