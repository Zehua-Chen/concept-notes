\chapter{Sampling}

Properties of \textbf{well-distributed} sampling (assuming sampling on
a unit square)
\begin{enumerate}
  \item Not structured, involves some kind of randomness
  \item Uniform distribution; no gaps
  \item Projections in 1D along x, y axis are uniform
  \item There is a non-trivial minimum distance between all sample points
\end{enumerate}

\section{Discrepancy}

  \begin{enumerate}
    \item Take a region $ S $
    \item Take a portion of $ S $, $ V $
    \item \textbf{Disrepancy} is the different between volume ratio of $ V:S $
    and the ratio of the number of points in $ V $ to the number of points in
    $ S $
  \end{enumerate}

\section{Aliasing}

  \newglossaryentry{aliasing}
  {
    name=aliasing,
    description={when a function has been reconstructed as a
    different function; especially when high frequency details are lost and
    represented as low frequency}
  }

  \begin{itemize}
    \item \textbf{\Gls{aliasing}}: \Glsdesc{aliasing};
    \item In a ray tracer, we are dividing images into squares and reconstruct
    a function that tells us how light is passing through the square
    \item Pixels have both discrete ($ d $) and continuous ($ c $) locations
    \begin{equation}
      c = d + \frac{1}{2}
    \end{equation}
    $ \left( 3, 5 \right) $ is located at $ \left( 3.5, 5.5 \right) $
  \end{itemize}

\section{Regular Sampling}

  \newglossaryentry{regular-sampling}
  {
    name=regular sampling,
    description={shoot rays through a $ n $ by $ n $ grid of subpixels.
    Color of a pixel is the average color of the subpixels}
  }

  \textbf{\Gls{regular-sampling}}: \glsdesc{regular-sampling}

  \begin{itemize}
    \item Expensive to perform
    \item Does not solve moire pattern
  \end{itemize}

\section{Random Sampling}

  \newglossaryentry{random-sampling}
  {
    name=random sampling,
    description={rather than shooting rays through a grid, shoot rays from
    random locations in the pixel}
  }

  \textbf{\Gls{random-sampling}}: \glsdesc{random-sampling}

  \begin{itemize}
    \item Reduce in noisy result
    \item People alias noisy to aliasing visually
  \end{itemize}

\section{Jittered Sampling}

  \newglossaryentry{jittered-sampling}
  {
    name=jittered sampling,
    description={create a grid and randomly generate a sample in each cell}
  }

  \textbf{\Gls{jittered-sampling}}: \glsdesc{jittered-sampling}

  \begin{itemize}
    \item Better than random, but still poorly distributed
  \end{itemize}

\section{n-rooks Sampling}

  \newglossaryentry{nrooks-sampling}
  {
    name=n-rooks sampling,
    description={create a grid but make sure each row and column are sampled}
  }

  \textbf{\Gls{nrooks-sampling}}: \glsdesc{nrooks-sampling}

  \begin{itemize}
    \item Good 1D distribution
    \item Poor 2D distribution
  \end{itemize}

\section{Multi-Jittered Sampling}

  \newglossaryentry{multi-jittered-sampling}
  {
    name=multi-jittered sampling,
    description={create a $ n \times n $ grid, known as \textbf{fine grid};
    then create a $ \sqrt{n} \times \sqrt{n} $ coarse grid each
    containing $ n $ fine grid's cells.
    Make sure each coarse grid cell contains one sample
    and the sample has unique fine grid row and column of the fine grid.}
  }

  \textbf{\Gls{multi-jittered-sampling}}: \glsdesc{multi-jittered-sampling}

  \begin{itemize}
    \item Good 1D projection from rock condition
    \item Good 2D projection from stratification
  \end{itemize}

\section{Hammersley Sampling}

  Deterministic quasi-random sequence

  \paragraph{1D} For a number $ i $, convert $ i $ to binary format, the
  $ \Phi_{x} \left( i \right) $ is the \textbf{radical inverse based $ x $ of $ i $}

  \begin{equation}
    \Phi_{2}\left( i \right)
      = \sum_{j = 0}^{n} a_{j} \left( i \right) 2^{-j - 1}
      = a_{0} \left( i \right) \frac{1}{2}
      + a_{1} \left( i \right) \frac{1}{4}
      + a_{2} \left( i \right) \frac{1}{8}
  \end{equation}

  $ a $s are the binary representations of $ i $, in this case (based 2).

  \paragraph{2D} We want to have $ n $ samples
  \begin{equation}
    p_{i}
      = \left( x_{i}, y_{i} \right)
      = \left[ \frac{i}{n}, \Phi_{2}\left( i \right) \right]
  \end{equation}

  \begin{itemize}
    \item Too structured; lead to artifacts in some applications
    \item For a given $ n $, only one sequence exists
  \end{itemize}

\section{Halton Sampling}

  \begin{equation}
    p_{i} =
    \left(
      \Phi_{2} \left( i \right),
      \Phi_{3} \left( i \right),
      \Phi_{5} \left( i \right)
    \right)
  \end{equation}
  Note that the bases are all prime numbers from small to big

  \begin{itemize}
    \item Generalizes well into n-d space
    \item Number of samples is not known ahead of times
  \end{itemize}

\section{Poisson Disk Sampling}

  $ r $ minimum distance between samples

  \paragraph{Algorithm}
  \begin{enumerate}
    \item Create a $ n $ dimensional background grid with cells of size
    $ \frac{r}{\sqrt{n}} $
    \item Keep a list of samples that has been generated
    \item Keep a list of indices of the samples that has been generated, aka.
    the \say{active list}
    \item Keep a list that represents the uniform grid; cells have two values
    \begin{itemize}
      \item $ -1 $: the cell have no samples
      \item $ i $: the cell has a sample. The value is the index of the sample
      in the samples list
    \end{itemize}

    \item Randomly, uniformly select a start sample and insert it into the
    \say{active list}
    \item While the \say{active list} is not empty
    \begin{enumerate}
      \item Select a random index $ i $ from the active list
      \item Generate up to $ k $ samples within distances of $ r, 2r $,
      around the sample; $ k = 30 $ is a popular choice
      \item For each sample, make sure it is not within distance $ r $ of
      existing samples (use the background grid)
      \item If a point is far enough from other points (previous step), add
      the point to the \say{active list}
      \item If after $ k $ attempts no such point is found, then remove $ i $
      from the active list
    \end{enumerate}
  \end{enumerate}

  \paragraph{Properties}
  \begin{itemize}
    \item Not actually a low discrepancy sequence
    \item Assure a minimum distance between samples
    \item Fast and generalizes well into higher dimensions
  \end{itemize}
