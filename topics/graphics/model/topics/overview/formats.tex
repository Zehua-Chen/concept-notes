\section{Formats}

\subsection{Storage Format}

  \begin{itemize}
    \item \textbf{STL}: faceset; the file is made of lines of floats; each line
    represent a triangle
    \begin{itemize}
      \item No connectivity
      \item \textbf{Storage} 36 bytes per triangle
      ($ 4 \times \left( 3 \times 3 \right) $); 72 bytes per vertex (number of
      vertices is 2 times the number of faces)
    \end{itemize}

    \item \textbf{OBJ}: indexed faceset; use two lists, one of vertices and
    one of faces (made of three indices into the list of vertices)
    \begin{itemize}
      \item No neighborhood information
      \item Changing individual vertices is more performant in \textbf{OBJ}
      than \textbf{STL}
      \item \textbf{Storage}: 12 bytes per face and vertex, 36 bytes per
      vertex $ 12 + 12 \times 2 $
    \end{itemize}

    \item \textbf{GLTF}
    \begin{itemize}
      \item Node hiearchy, PBR material textures and camera
    \end{itemize}

    \item \textbf{BIN}
    \begin{itemize}
      \item Geometry: vertices and indices
      \item Animation:  key frames
      \item Skins: inverse bind matrices
    \end{itemize}

    \item \textbf{Half Edge}\footnote{two half edges of opposite direction
    exists on top of a real edge}: the file is made of three lists
    \begin{itemize}
      \item \textbf{Half Edge List}:
      \begin{itemize}
        \item Vertex ref
        \item Face ref
        \item Next half edge ref
        \item Previous half edge ref (can be optimized away)
        \item Opposite half edge ref (can be optimized away)
      \end{itemize}

      \item \textbf{Face List}
      \begin{itemize}
        \item Half edge ref
      \end{itemize}

      \item \textbf{Vertex List}
      \begin{itemize}
        \item Position
        \item Half edge ref
      \end{itemize}

      \item \textbf{Space}: 16 bytes per vertex, 4 bytes per face, 12 bytes
      per half edge
      \begin{equation}
        \sum = 16 + 4 \times 2 + 12 \times 6
      \end{equation}

      Total space is 96 bytes per vertex; if \textbf{previous and opposite
      half edges are stored}, then 144 bytes per vertex

      \item Can be used to traverse vertex neighborhoods; used more in
      computational modeling
    \end{itemize}
  \end{itemize}

\subsection{Graphics API Format}

  \begin{itemize}
    \item \textbf{Points}: every $ 3 $ floats represents the $ x, y, z $ of
    a point
    \item \textbf{Lines}: every $ 2 \times 3 $ represents the start point
    and end point of a line
    \item \textbf{Strips}: every point form a line with its neighbors;
    producing a path (\textbf{does not draw path between the last three float
    and the first three float}, aka no closing)
    \item \textbf{Line Loop}: strips + extra line from the last three to the
    first three
    \item \textbf{Triangle}: $ 3 \times 3 $ represent the three vertices of a
    triangle
    \item \textbf{Triangle Strips}: every $ 3 $ floats (except the first two in
    buffer) form a triangle with the previous two $ 3 $ floats
    \item \textbf{Triangle Fan}: every $ 3 $ floats (except the first two)
    form a triangle by connecting to the previous $ 3 $ floats and the first
    $ 3 $ floats
  \end{itemize}

  \subsubsection{T Vertices}

    T vertices are vertices that sits on a neighboring triangle's edge

    \begin{itemize}
      \item Produce visual artifacts during shading
      \item Typically avoided by using modeling tools
    \end{itemize}
