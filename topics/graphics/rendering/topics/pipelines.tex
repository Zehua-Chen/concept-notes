\chapter{Pipelines}

\section{Transformation Pipeline}

  Transforming model coordinates to viewport coordinates

  \begin{equation}
    \begin{bmatrix}
      x_{s} \\
      y_{s} \\
      0 \\
      1
    \end{bmatrix}
    =
    \begin{bmatrix}
      \text{W2V}
    \end{bmatrix}
    \begin{bmatrix}
      \text{Projection}
    \end{bmatrix}
    \begin{bmatrix}
      \text{View}
    \end{bmatrix}
    \begin{bmatrix}
      \text{Model}
    \end{bmatrix}
    \begin{bmatrix}
      x_{m} \\
      y_{m} \\
      z_{m} \\
      1
    \end{bmatrix}
  \end{equation}

  \begin{enumerate}
    \item The \textbf{model matrix} map local object coordinates to
    world coordinates (ex. rotation)
    \item Given a camera looking down the z-axis from the origin,
    the \textbf{view matrix} map world coordinates to camera coordinates
    \item The \textbf{projection matrix} map camera coordinates to canonical
    view volume
    \item The \textbf{W2V (window to view) matrix} map canonical view volume
    to screen space
  \end{enumerate}

  Every transformations except W2V transformation happen in vertex
  processing stage

  \subsection{In Unity}

    \begin{itemize}
      \item Transform component on a camera is the view matrix
      \item Camera settings are used to create
      \begin{itemize}
        \item Projection matrix
        \item Viewport matrix
      \end{itemize}
    \end{itemize}

  \subsection{Projection}

    Projection refers to the process of reducing dimensionality. There are two
    modes of projection

    \begin{itemize}
      \item \textbf{Orthographic}: toss out the z component
      \item \textbf{Perspective}: scale decreases as distance increases
    \end{itemize}

    Below is on perspective projection:

    \begin{align}
      \frac{y_{\text{clip}}}{d} &= \frac{y_{\text{view}}}{-z_{\text{view}}} \\
      y_{\text{clip}}
      &= d\frac{y_ {\text{view}}}{-z_{\text{view}}} \\
      &= \frac{y_{\text{view}}}{-z_{\text{view}} / d}
    \end{align}

    \begin{itemize}
      \item The equation applies in situations where
      \begin{itemize}
        \item The \textbf{eyes} are \textbf{centered at the origin}
        \item The \textbf{view} is in \textbf{negative z direction}
        \item The clip is between the eyes and the view
      \end{itemize}

      \item $ d $: distance from eyes to the screen (clip)
      \item This equation also applies to $ x $ and $ y $
    \end{itemize}

\section{Graphics Pipeline}

  \begin{enumerate}
    \item \textbf{Vertex}
    \begin{itemize}
      \item The majority of transformation pipeline happens here,
      \textbf{except W2V}
    \end{itemize}
    \item \textbf{Rasterization}
    \item \textbf{Fragment}
    \begin{itemize}
      \item Not pixels: each fragment has a 2D location in a raster and a color;
      final value is found by appliying hidden surface removal and possibly
      compositing to a set of fragments
    \end{itemize}
  \end{enumerate}
