\chapter{Model}

\begin{itemize}
  \item Models are defined in terms of \textbf{primitives (i.e. triangles)}
  \begin{equation}
    \left(
      \left( x_{1}, y_{1}, z_{1} \right),
      \left( x_{2}, y_{2}, z_{2} \right),
      \left( x_{3}, y_{3}, z_{3} \right)
    \right)
  \end{equation}

  \begin{itemize}
    \item Vertices are the coordinates within the triangle
  \end{itemize}

  \item Triangle form meshes
  \item \textbf{Data structures}
  \begin{itemize}
    \item Meshes needs to be stored in an efficient way so that modifying
    vertices (ex. rotation) can be done efficiently
    \item \textbf{Common data structures}:
    \begin{itemize}
      \item Doubly connected edge list
      \item Half-edge data structure
    \end{itemize}
  \end{itemize}

  \item \textbf{Coherent model}: a model wehre the inside and the outside is
  completely isolated; if we fill the model with gas, the gas would not leak
  \begin{itemize}
    \item The triangles in the model fit perfectly
    \begin{itemize}
      \item Each edge has two neighbors of triangles
      \item No triangles intersect unless they are adjacent
    \end{itemize}

    \item Such models are required if inside and outside is critical to VWG
    \item 3D modeling tools automatically construct coherent models
  \end{itemize}

  \item Why triangles
  \begin{itemize}
    \item Algorithms are easy to be made
    \item GPUs prefer smaller primitives as they are easier to parallelly
    process
  \end{itemize}

  \item \textbf{Stationary vs movable models}
  \begin{itemize}
    \item \textbf{Stationary}: same coordiantes forever
    \item \textbf{Movable}: can be transformed into various positions and
    orientations
  \end{itemize}
\end{itemize}

\import{./}{formats}
\import{./}{coordinate}
