\section{Formats}

\subsection{Storage Format}

  \begin{itemize}
    \item \textbf{STL}
    \begin{itemize}
      \item Aka. faceset
      \item A file of lines of floats
      \item Each line represent a triangle
    \end{itemize}

    \item \textbf{OBJ}
    \begin{itemize}
      \item Aka. indexed faceset
      \item Use two columns, one of vertices and one of indexes (3x)
      \item The indexes refer to the index in the vertices columns;
      three vertices represent a triangle
      \item Changing individual vertices is more performant in \textbf{OBJ}
      than \textbf{STL}
    \end{itemize}

    \item \textbf{GLTF}
    \begin{itemize}
      \item Node hiearchy, PBR material textures and camera
    \end{itemize}

    \item \textbf{BIN}
    \begin{itemize}
      \item Geometry: vertices and indices
      \item Animation:  key frames
      \item SKins: inverse bind matrices
    \end{itemize}
  \end{itemize}

\subsection{Graphics API Format}

  \begin{itemize}
    \item \textbf{Points}: every $ 3 $ floats represents the $ x, y, z $ of
    a point
    \item \textbf{Lines}: every $ 2 \times 3 $ represents the start point
    and end point of a line
    \item \textbf{Strips}: every point form a line with its neighbors;
    producing a path (\textbf{does not draw path between the last three float
    and the first three float}, aka no closing)
    \item \textbf{Line Loop}: strips + extra line from the last three to the
    first three
    \item \textbf{Triangle}: $ 3 \times 3 $ represent the three vertices of a
    triangle
    \item \textbf{Triangle Strips}: every $ 3 $ floats (except the first two in
    buffer) form a triangle with the previous two $ 3 $ floats
    \item \textbf{Triangle Fan}: every $ 3 $ floats (except the first two)
    form a triangle by connecting to the previous $ 3 $ floats and the first
    $ 3 $ floats
  \end{itemize}

  \subsubsection{T Vertices}

    T vertices are vertices that sits on a neighboring triangle's edge

    \begin{itemize}
      \item Produce visual artifacts during shading
      \item Typically avoided by using modeling tools
    \end{itemize}
