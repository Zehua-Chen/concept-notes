\chapter{Model}\label{chapter: model}

Models are defined in terms of \textbf{primitives (i.e. triangles)}

\begin{equation}
  \left(
    \left( x_{1}, y_{1}, z_{1} \right),
    \left( x_{2}, y_{2}, z_{2} \right),
    \left( x_{3}, y_{3}, z_{3} \right)
  \right)
\end{equation}

\begin{itemize}
  \item Vertices are the points of triangles
  \item Triangle form meshes
  \begin{itemize}
    \item \textbf{Surface Normals}: each triangle has an outward surface
    \textbf{normal} \footnote{normals are vectors that are perpendicular to a
    surface}
  \end{itemize}
  \item \textbf{Data structures}: doubly connected edge list, aka, half-edge
  data structure

  \item \textbf{Coherent model}: a model wehre the inside and the outside is
  completely isolated; if we fill the model with gas, the gas would not leak
  \begin{itemize}
    \item The triangles in the model fit perfectly
    \begin{itemize}
      \item Each edge has two neighbors of triangles
      \item No triangles intersect unless they are adjacent
    \end{itemize}

    \item Such models are required if inside and outside is critical to VWG
    \item 3D modeling tools automatically construct coherent models
  \end{itemize}

  \item Why triangles
  \begin{itemize}
    \item Algorithms are easy to be made
    \item GPUs prefer smaller primitives as they are easier to parallelly
    process
  \end{itemize}

  \item \textbf{Stationary vs movable models}
  \begin{itemize}
    \item \textbf{Stationary}: same coordiantes forever
    \item \textbf{Movable}: can be transformed into various positions and
    orientations
  \end{itemize}

  \item \textbf{Viewing the Model}
  \begin{itemize}
    \item Mapping virtual points to display points
    \item Determine what parts of the model should look considering light and
    surface characeteristics
  \end{itemize}
\end{itemize}

\section{Mesh Properties}

  \begin{itemize}
    \item \textbf{Single component}: only one mesh entity
    \item \textbf{Multiple component}: multiple mesh entities
    \item \textbf{Closed}: no holes in meshes
    \item \textbf{With boundaries}: some edges border with only one triangle
    \textbf{boundary edge}
    \item \textbf{Not only triangles}: shapes other than triangles
    \item \textbf{Orientability}:
    \begin{itemize}
      \item On an \textbf{orientable} object, when a surface slides around the
      face, and reaches it starting point, the surface would look the same
      \begin{itemize}
        \item Allows us to specify a set of normals, which is important for
        shading
      \end{itemize}

      \item On an \textbf{non-oriental} object, the surface would not
      look the same when it comes back the starting point; no consistent
      definition of clock-wise and counter-clockwise
    \end{itemize}

    \item \textbf{Manifold}
    \begin{enumerate}
      \item Every edge connects to exactly two faces
      \item Vertex neighborhood is \textbf{disk-like}\footnote{when
      you crush the vertex on the mesh, you can draw a circle around the
      vertex}
      \begin{itemize}
        \item Ex. cube is manifold
        \item Ex. two cubes joined by a vertex is not manifold
      \end{itemize}
    \end{enumerate}

    \item \textbf{Watertight}: orientable + manifold
    \item \textbf{Ordered}: vertices (of triangle) in counter-clockwise
    when viewed from normal
    \item \textbf{Number of Genus} on a surface
    \begin{itemize}
      \item Sphere has a genus 0
      \item Circle has a genus 1
    \end{itemize}
  \end{itemize}

  \subsection{Euler Characteristic}

    For a \textbf{closed (no boundaries)}, manifold, connected surface mesh

    \begin{align}
      V - E + F &= 2 \left( 1 - G \right) \\
      &= \chi
    \end{align}

    \begin{enumerate}
      \item $ V $: number of vertices
      \item $ E $: number of edges
      \begin{equation}
        E = \frac{3F}{2}
      \end{equation}

      \item $ F $: number of faces
      \item $ G $: genus
      \item $ 2 \left( 1 - G \right) $: the characteristic
      \begin{itemize}
        \item Sphere has a euler characteristic of $ 2 $
      \end{itemize}
    \end{enumerate}

    From euler characteristic, we can derive \textbf{triangle mesh statistics}
    (only applies to triangle meshes)

    \begin{align}
      E &\approx 3V \\
      F &\approx 2V
    \end{align}

\import{./}{formats}
