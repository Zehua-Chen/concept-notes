\section{Rasterization}

  After knowing the colors of each vertex, we need to know how to fill the
  inside of the triangles

\subsection{Linear Interpolation}

  Interpolation refers to the process of knowing a few points on the function
  and then modeling the rest of the function. In rasterization, we knows
  the colors on the vertices and we now want to know how to color the inside
  of the triangles

  \begin{align*}
    l\left( t \right) &=
      \left( 1 - t \right) \left< x_{0}, y_{0} \right>
      + t \left< x_{1}, y_{1} \right> \\
    f\left( t \right) &=
      \left( 1 - t \right) f\left( x_{0}, y_{0} \right)
      + t f\left( x_{1}, y_{1} \right)
  \end{align*}

  \begin{itemize}
    \item $ l $: gives the coordinate of a point that is $ t $ distance
    from $ \left< x_{0}, y_{0} \right> $
    \item $ f $: gives the value on the coordinate that is $ t $ distance
    from $ \left< x_{0}, y_{0} \right> $
  \end{itemize}

\subsection{Barycentric Coordinates}

  One way that the GPU interpolates color on the face of a triangle is
  implemented using \say{barycentric coordinates}

  \begin{itemize}
    \item Three points
    \begin{itemize}
      \item $ \left< 1, 0, 0 \right> $
      \item $ \left< 0, 1, 0 \right> $
      \item $ \left< 0, 0, 1 \right> $
    \end{itemize}
    represent the three vertices of a triangle.
  \end{itemize}

  \subsubsection{Constructing Barycentric Coordinates}

    \begin{itemize}
      \item Given a point $ p $, $ p $ and the three vertices of a triangle
      $ p_{0}, p_{1}, p_{2} $, $ p, p_{0}, p_{1}, p_{2} $ forms three
      smaller triangles.

      \begin{align}
        a_{0}\left( p \right) &= A\left( p_{1}, p_{2}, p \right) \\
        a_{1}\left( p \right) &= A\left( p_{2}, p_{0}, p \right) \\
        a_{2}\left( p \right) &= A\left( p_{0}, p_{1}, p \right)
      \end{align}

      From the areas of the smaller triangles, the barycentric coordinate
      of the point $ p $ can then be calculated

      \begin{align}
        b_{0} &= \frac{a_{0}}{A\left( p_{0}, p_{1}, p_{2} \right)} \\
        b_{1} &= \frac{a_{1}}{A\left( p_{0}, p_{1}, p_{2} \right)} \\
        b_{2} &= \frac{a_{2}}{A\left( p_{0}, p_{1}, p_{2} \right)} \\
        b_{0} + b_{1} + b_{2} &= 1 \\
        P &= \left< b_{0}, b_{1}, b_{2} \right>
      \end{align}

      \item Barycentric coordinate construction is similar to linear
      interpolation
      \item There is \textbf{no restriction on the dimension of space} the
      triangle was orignally in
    \end{itemize}

  \subsubsection{Point Testing}

    \begin{enumerate}
      \item Convert the point of interest to barycentric coordinate
      \item If one component of the converted point is negative, the point is
      outside of the triangle
    \end{enumerate}

  \subsubsection{Interpolate Colors}

    Given a barycentric coordinate
    $ p = \left< \lambda_{0}, \lambda_{1}, \lambda_{2} \right> $, and the
    colors at the corners,
    $ f\left( a \right), f\left( b \right), f\left( c \right) $. The color
    at $ p $, $ f\left( p \right) $ is

    \begin{equation}
      f\left( p \right)
        = \lambda_{0} f\left( a \right)
        + \lambda_{1} f\left( b \right)
        + \lambda_{2} f\left( c \right)
    \end{equation}
