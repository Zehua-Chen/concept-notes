\section{Phong Reflection}

\begin{equation}
  I_{p} = k_{a} i_{a} +
  \sum_{m \in \text{lights}}
  \left(
    k_{d} \left( \hat{L}_{m} \cdot \hat{N} \right) i_{m, d} +
    k_{s} \left( \hat{R}_{m} \cdot \hat{V} \right)^{\alpha} i_{m, s}
  \right)
\end{equation}

\begin{itemize}
  \item All vectors are unit vectors
  \item \textbf{Ambient Term}:
  \begin{equation}
    k_{a} i_{a}
  \end{equation}
  \begin{itemize}
    \item \textbf{Light source}: indirect light
  \end{itemize}

  \item \textbf{Diffuse Term}:
  \begin{equation}
    k_{d} \left( \hat{L}_{m} \cdot \hat{N} \right) i_{m, d}
  \end{equation}
  \begin{itemize}
    \item $ k_{d} $: material reflectivity
    \item $ i_{m, d} $: $ \left( r, g, b \right) $ for a single light $ m $
    \item $ \hat{L}_{m} \cdot \hat{N} $: the angle between $ L $ and $ N $
    (which are unit vectors)
    \item Diffuse reflection is strongest when $ L $ is
    perpendicular to $ N $
    \item Models an idealized\footnote{scatter light uniformly} surface
    (aka, rough surfaces)
  \end{itemize}

  \item \textbf{Specular Term}
  \begin{equation}
    k_{s} \left( \hat{R}_{m} \cdot \hat{V} \right)^{\alpha} i_{m, s}
  \end{equation}
  \begin{itemize}
    \item $ \hat{R}_{m} \cdot \hat{V} $: the angle between reflection and
    light source
    \item Specular terms models shiny surfaces
    \item \textbf{Shinier}: a higher $ \alpha $; \textbf{rougher}:
    lower $ \alpha $
  \end{itemize}
\end{itemize}

\subsection{Open Questions}

  \begin{itemize}
    \item \textbf{Include attenuation
    \footnote{reduction in illumination as distance increases} in equation}:
    Add a term in the front of summation
    \begin{equation*}
      \frac{1}{ad^{2} + bd + c}
    \end{equation*}

    \item \textbf{When there are no specular surfaces, no moving light}:
    colors of vertices can be optimized to be precomputed at the start and
    be reused later on

    \item \textbf{Modeling 3 wavelength is sufficient} because most people
    have 3 tones in eyes (the three tones are similar to R, G, B)
  \end{itemize}

\subsection{Blinn Phong Model}

  \begin{equation}
    H = \frac{L + V}{\left| L + V \right|}
  \end{equation}

  \begin{itemize}
    \item $ L $: the light
    \item $ V $: the viewer
  \end{itemize}

  In Blinn Phong model, we use $ H $ (the halfway vector) instead of $ r $

  \begin{itemize}
    \item Slightly cheaper than the Phong model
    \item Claimed to be more physically correct
  \end{itemize}
