\section{Implementations}

\subsection{Linear Probing, Double Hashing}

  \begin{itemize}
    \item Both hash collision strategy requires the following member 
    variables to store key and values:
    \begin{enumerate}
      \item An array of \highlight{key-value pairs 
      with \textbf{optional} semnatics}
      \footnote{Optional semantics: the value is either empty or hold 
      valid values};
      
      \item An \highlight{array of booleans} used to 
      \highlight{determine if probing should be performed};
    \end{enumerate}
  \end{itemize}
  
  \paragraph{Why an Array of Booleans?}
  \begin{itemize}
    \item The array is used to determine if a value is 
    either \highlight{occupying} or \highlight{was previously 
    occupying} a cell;
    
    \item Consider the following case:
    \begin{enumerate}
      \item There is an entry in the hashtable at index 0;
      \item A newly inserted entry gives an initial hash value of 0.
      After computing linear probing or double hasing,
      the new entry is placed in index 1;
      \item Remove the entry at 0 by invalidating the cell at index 0;
      \item A if-key-exits operation is invoked with the key of the 
      entry at index 1 being what the operation is looking for;
      \begin{enumerate}
        \item The initial hashing operation output index 0;
        \item Without the boolean array, the lookup will return $ -1 $ as
        the cell at index 0 is invalid;
        \item With the boolean array, the lookup will continue to index $ 1 $,
        as there was a cell that was previously occupying index 0.
      \end{enumerate}
    \end{enumerate}
  \end{itemize}