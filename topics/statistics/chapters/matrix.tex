\chapter{Matrix}

\section{Vectors}

  \subsection{Orthognol Vectors}

    Multiple vectors are said to be orthognol if
    \begin{itemize}
      \item \textbf{Mathmetically}: the vectors have a dot product of 0;
      \item \textbf{Visually}: the vectors are perpendicular to each other;
    \end{itemize}
    
  \subsection{Unit Vectors}
  
    \begin{itemize}
      \item A unit vector is a vector \textbf{that can fit inside a unit sphere};
    \end{itemize}

\section{Identity Matrix}

  \begin{equation}
    I = 
    \begin{bmatrix}
      1 & 0 & 0 \\ 
      0 & 1 & 0 \\
      0 & 0 & 1
    \end{bmatrix}
  \end{equation}
  
\section{Row Reduction}
  
\section{Determinant}

  Given a 2x2 matrix $ A $:
  \begin{displaymath}
    A = 
    \begin{bmatrix}
      a & b \\
      c & d 
    \end{bmatrix}
  \end{displaymath}
  the determinant of $ A $ is $ \left| A \right| $
  \begin{equation}
    \left| A \right| = ad - bc 
  \end{equation}
  
\section{Eigen Valus and Eigen Vectors}

  \begin{align}
    Av &= \lambda v \\
    \left| A - \lambda I \right| &= 0 \\
    \left( A - I \right)v &= 0
  \end{align}
  
  \begin{itemize}
    \item $ \lambda $: \textbf{eigen value};
    \item $ v $: \textbf{eigen vector} for $ \lambda $
    \begin{itemize}
      \item $ kv $ is also an \textbf{eigen vector} for $ \lambda $
    \end{itemize}
  \end{itemize}
  
  \subsection{Characteristic Equation}
  
    \begin{align*}
      A &= 
      \begin{bmatrix}
        5 & 3 \\
        3 & 5
      \end{bmatrix} 
      \\
      \left| A - \lambda I \right| &= 0 \\
      \begin{bmatrix}
        5 - \lambda & 3 \\
        3 & 5 - \lambda
      \end{bmatrix} &= 0 
      \\
      \left( 5 - \lambda \right)^{2} - 3^{2} &= 9 \\
      \lambda &= 8, 2
    \end{align*}
    
\section{Transpose}

  \begin{equation}
    \begin{bmatrix}
      1 & 2 & 3 \\
      4 & 5 & 6 \\
      7 & 8 & 9
    \end{bmatrix}^{T}
    = 
    \begin{bmatrix}
      1 & 4 & 7 \\
      2 & 5 & 8 \\
      3 & 6 & 9
    \end{bmatrix}
  \end{equation}

  \begin{itemize}
    \item A transpose of a matrix is a new matrix in which \textbf{original colums are now the rows} and \textbf{original rows are now the columns};
  \end{itemize}
  
\section{Diagnolization}

  \begin{itemize}
    \item A diagnolized matrix is a matrix in which there are only non-zero values along the diagnol of the matrix, and the rest of the matrix are all zeros;
    \item To diagnolize a matrix:
    \begin{enumerate}
      \item Find the \textbf{eigen values and eigen vectors};
      \item Conver the \textbf{eigen vectors into unit vectors};
      \item Sort the eigen values in descending order;
      \item Compose a matrix $ U $ that is composed of eigen vectors arranged in a way where their column index in the matrix matches the index of their corresponding eigen value in the sorted array of eigen values;
      \item $ \Lambda $ is the diagnolized matrix;
      \begin{equation}
        \Lambda = U^{T} A U
      \end{equation}
    \end{enumerate}
  \end{itemize}
  
\section{Rotation}

  \begin{itemize}
    \item To rotate a matrix, multiply a matrix by a \textbf{orthognol matrix};
  \end{itemize}
