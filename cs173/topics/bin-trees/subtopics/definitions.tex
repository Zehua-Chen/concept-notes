\section{Definitions}

\subsection{Type of Trees}

  \subsubsection{Binary Trees}

    \begin{equation}
      T = 
      \begin{cases}
        \{ r, T_{l}, T_{r} \} \\ 
        \{ \} = \varnothing
      \end{cases}
    \end{equation}

  \subsubsection{Full Trees}

    \begin{equation}
      F = 
      \begin{cases}
        \{ r, T_{l}, T_{r} \} \\ 
        \{ \} = \varnothing
      \end{cases}
    \end{equation}
    
    \begin{itemize}
      \item $ T_{l}, T_{r} $ both have either \highlight{0 or 2 children};
    \end{itemize}

  \subsubsection{Perfect Trees}

    \begin{equation}
      \begin{cases}
        P_{-1} = \{ \} \\ 
        P_{h} = \{ r, P_{h - 1}, P_{h - 1} \}
      \end{cases}
    \end{equation}
    
    \begin{itemize}
      \item A perfect tree is defined in terms of its height;
      \item The \highlight{children of each node has the same height};
    \end{itemize}

  \subsubsection{Complete Trees}

    \begin{equation}
      \begin{cases}
        C_{-1} = \{ \} \\ 
        C_{h} = 
        \begin{cases}
          \{r, C_{h - 1}, P_{h - 2} \} \\ 
          \{r, P_{h - 1}, C_{h - 1} \} \\ 
        \end{cases}
      \end{cases}
    \end{equation}
    
    \begin{definition}
      \textbf{Complete tree}: a \highlight{perfect tree for every level 
      except the last level}, where \highlight{the level is pushed to the left};
    \end{definition}
  
\subsection{BST Tree Variants}

  \subsubsection{Height Balance}

    \begin{equation}
      b = \height \left( T_{r} \right) - \height \left( T_{l} \right)
    \end{equation}
    
    \begin{equation}
      b = 
      \begin{cases}
        < 0, \text{heavy on the left side} \\ 
        > 0, \text{heavy on the right side}
      \end{cases}
    \end{equation}
    
    \begin{definition}
      To say a tree is balanced is to say 
      \begin{equation}
        \left| b \right| \leq 1
      \end{equation}
    \end{definition}
    
  \subsubsection{AVL Tree}

    \begin{definition}
      An AVL Tree is similar to a BST Tree, except that the 
      \highlight{insert or remove} code balance the tree along path of 
      insertion or removal so that the $ \left| b \right| $ of each node 
      is $ < 1 $. 
    \end{definition}
    
  \subsubsection{KD Trees}

    \begin{definition}
      A KD Tree is very similar to a AVL tree, except that 
      \highlight{each level represents a \textbf{dimension}}. When nodes 
      are inserted, or searched, the code 
      \highlight{decides to go left or right based on comparison 
      on the datas at a specific dimension, of \enquote{this} node and the query or insertion}.
    \end{definition}