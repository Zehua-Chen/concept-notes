\documentclass{note}

\usepackage{float}
%\usepackage{color, colortbl}
%\usepackage{longtable}
\usepackage{tabu}
\usepackage{color}
\usepackage{listings}

\definecolor{codebackground}{rgb}{0.9,0.9,0.9}
\definecolor{codenumber}{rgb}{0.7,0.7,0.7}
\definecolor{codekeyword}{rgb}{0,0,1}
\definecolor{codestring}{rgb}{1,0,0}
\lstdefinestyle{cpp-list}{
    backgroundcolor=\color{codebackground},
    numberstyle=\color{codenumber},
    keywordstyle=\color{codekeyword},
    stringstyle=\color{codestring},
    numbers=left,
    caption=C++ Implementation of Hypercubes Edges,
}
\lstset{style=cpp-list}

% For ceil and floor
\usepackage{mathtools}
\DeclarePairedDelimiter\floor{\lfloor}{\rfloor}
\DeclarePairedDelimiter\ceil{\lceil}{\rceil}

\subject{CS 173}
\date{March 6, 107}
\id{CS17310703051}
\title{Recursive Definition}

\begin{document}
\begin{note}{Exam 1}

\section{Recursive Definition}

\begin{itemize}
    \item Similar to that in programming languages;
    \item A recursive definition has two parts:
    \begin{itemize}
        \item \hl{At least one} \textbf{Base case};
        \begin{itemize}
            \item Multiple base cases do no harm and may help with constructing the proof;
        \end{itemize}
        \item \textbf{Recursive formula};
    \end{itemize}
    \item These two components do not have to be labeled.
\end{itemize}

\paragraph{Example}

\begin{itemize}
    \item $ F_{0} = 0 $;
    \item $ F_{1} = 1 $;
    \item $ F_{i} = F_{i - 1} + F_{i - 2}, \forall i > 2 $;
\end{itemize}

    \subsection{Closed Form}

    \begin{itemize}
        \item A \textbf{closed form} is a \textbf{expression} that \hl{does not involve}:
        \begin{itemize}
            \item \textbf{Recursion};
            \item \textbf{Summation};
        \end{itemize}
        \item \hl{Closed form should have a input variable} as recursive definitions have \textbf{base cases};
    \end{itemize}

    \paragraph{Unrolling}: Keeping writing an expression in terms of itself. For example, given $ T(1) = 1, T(x) = 2 \times T(x - 1) - 1 $.
    The unrolling result will be:
    \begin{align*}
        T(x) &= 2 \times T(x - 1) - 1 \\
        &= 2 \times \left( 2 \times T(x - 2) - 1 \right) - 1\\
        &= ...
    \end{align*}

    \begin{itemize}
        \item \hl{Variables \textbf{outside of the recursive function calls} needs to be updated in each unroll as well};
        \begin{align*}
            T(x) &= T \left( \frac{x}{2} \right) + x  \\
            &= \left( T \left( \frac{x}{4} \right) + \frac{x}{2} \right) + x
        \end{align*}
    \end{itemize}

\section{Solving Problems}

    \subsection{Divide and Conquer}

    \paragraph{Divide and Conquer} the way of solving problems by dividing a big problem of size $ n $, into smaller sub problems,
    each of size $ \frac{n}{b} $

    \begin{align*}
        S(1) &= c\\
        S(n) &= a S\left(\ceil*{ \frac{n}{b}} \right) + f \left(n \right), \forall n \geq 2
    \end{align*}

    \begin{itemize}
        \item $ c $: the work needed to solve the base case;
        \item \hl{$ a $: the amount of sub problems};
        \item \hl{$ b $: the ratio of the size of a problem to that of its sub problem}.
        \item $ f \left(n \right) $: the work needed to divide a big problem into smaller problems and, or merging the solutions from smaller
        problems;
        \item $ \ceil*{} $ needs to be applied to ensure what feeds into $ S (n) $ is always an integer;
    \end{itemize}

    \subsection{Hypercubes}

    $ Q_{n} $ is a graph of corners and edges of an $ n $ dimensional cube.

    \begin{align*}
        Q_{o} &\text{ is a single nodes without edges;}\\
        Q_{n} &\text{ consists of two } Q_{n - 1} \text{ nodes with edges joining corresponding nodes;}
    \end{align*}

    \begin{itemize}
        \item $ Q_{n} $ has $ 2^{n} $ nodes;
        \item \hl{$ Q_{n} $ has a \textbf{diameter} of $ n $};
        \item $ Q_{n} $ has the following amount of edges:
        \begin{align}
            E(n) &= 2 E(n - 1) + V(n - 1)\\
            &= 2^{n - 1} \cdot n
        \end{align}
    \end{itemize}

        \subsubsection{C++ Code}
        \texttt{\lstinputlisting[language=C++]{hypercubes-edges.cpp}}

\end{note}
\end{document}