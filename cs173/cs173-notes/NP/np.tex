\documentclass{note}

\usepackage{float}
\usepackage{color, colortbl}
\usepackage{longtable}
\usepackage{tabu}
\usepackage[english]{babel}

\definecolor{red}{rgb}{1, 0, 0}

% For ceil and floor
\usepackage{mathtools}
\DeclarePairedDelimiter\floor{\lfloor}{\rfloor}
\DeclarePairedDelimiter\ceil{\lceil}{\rceil}

\newtheorem{definition}{Definition}

\subject{CS 173}
\date{April 2 -- 3, 107}
\id{CS17310704021}
\title{NP}

\begin{document}
\begin{note}{Exam 9}

\section{Multiplying Big Integers}

\begin{align*}
    T \left( 1 \right) &= d \\
    T \left( n \right) &= 3 T \left( \frac{n}{2} \right) + c \times n
\end{align*}

\begin{itemize}
    \item Anatolii Karatsuba's algorithm of multiplying numbers has a running time of:
    \begin{equation}
        O \left( n^{ \log_{2} 3 } \right)
    \end{equation}

    \item The original algorithm for multiplying numbers is $ O \left( n^{ 2 } \right) $
\end{itemize}

\section{Types of Problems}

    \subsection{EXP}

    \begin{itemize}
        \item \textbf{EXP}: problems that has to be solved using \textbf{exponential} time algorithm, proven. Examples are as follows:
        \begin{enumerate}
            \item Tower of Hanoi $ \Theta \left( 2^{n} \right) $;
            \begin{align*}
                T (1) &= d \\
                T (n) &= 2 T (n - 1) + c 
            \end{align*}
            \item Parse Tree
        \end{enumerate}
    \end{itemize}

    \subsection{NP}

    \begin{itemize}
        \item \textbf{NP}: problems that have existing \textbf{exponential} time algorithm. \hl{However, no one has proven them being 
        impossible to do using \textbf{polynomial time algorithm}}.
        
        \begin{enumerate}
            \item Graph Colorability Problems;
            \item Satisfiability of Boolean Circuits;
        \end{enumerate}

        \item Problems for which we can \hl{provide \textbf{succint positive} answers} are in \textbf{NP};
        \begin{itemize}
            \item For an answer to be succint, the result must be \hl{checkable in \textbf{polynomial time}};
        \end{itemize}
    \end{itemize}

        \subsubsection{Satisfiability of Boolean Circuits}

        \begin{itemize}
            \item \textbf{Circuit Satisfiability Problems}: problems where we find \hl{if there is a set of values $ n $ 
            that would create a fordbidden pattern of output}:
            \begin{itemize}
                \item To test, \hl{generate $ 2^{n} $ possbile input patterns} give $ n $ input wires;
                \item To justify the solution of the problem, \hl{show on pattern of input that will generate the fordbidden output};
                \item \hl{Therefore the problem is in \textbf{NP}};
            \end{itemize}
        \end{itemize}

        \paragraph{What is Boolean Circuit}

        \begin{itemize}
            \item A boolean circuit is a graph that consists of set of gates connected by wires;
            \item Each wires carries true (1) or false (0);
            \item The label on a gate determines how the value on its output value are determined by its input value:
            \begin{itemize}
                \item \textbf{Example}, for an output value of an \textbf{AND} gate to be true, both of its input wires needs to be true;
            \end{itemize}
            \item Labels on gates (nodes) are the logical operators;
        \end{itemize}

    \subsection{NP-Complete}

    \begin{itemize}
        \item \textbf{NP-Complete}: problems for \hl{which once \textbf{polynomial runtime} can be proven, all other NP problems can have
        polynomial runtime}; back-bone problems;
        
        \item \textbf{Examples}:
        \begin{itemize}
            \item Propositional logic satisfiability
            \item Clique
            \item The Marker Making problem
            \item Vertex cover
            \item The Travelling Salesman Problem
        \end{itemize}

        \item \hl{Problems typically remain \textbf{NP-complete} even when stripped to their base cases (extreme cases)}:
        \begin{itemize}
            \item Restrict boolean circuit to have a single output;
        \end{itemize}
    \end{itemize}

    \subsection{co-NP}

    \begin{itemize}
        \item Problems for which we can \hl{provide \textbf{succint negative} answers} are in \textbf{co-NP}:
        \begin{itemize}
            \item Circuit Safety
        \end{itemize}

        \item If both succint \textbf{positive and negative} answers can be provided, then the problem is in both \textbf{NP and co-NP};
    \end{itemize}

        \subsubsection{Circuit Safety}

        \begin{itemize}
            \item Prove none of the input patterns can provide forbidden output pattern;
            \item The problem is in \textbf{co-NP};
            \begin{itemize}
                \item There are \textbf{exponential amount of input pattern};
                \item Negative answers can be proven using \textbf{polynomial runtime};
            \end{itemize}
        \end{itemize}


    \subsection{P}

    \begin{itemize}
        \item Problems in \textbf{P} \hl{can be solved \textbf{in polynomial time}};
        \item \textbf{P} \hl{is a subset of \textbf{NP and co-NP}};
    \end{itemize}

\end{note}
\end{document}
